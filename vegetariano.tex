% This is the 'temporary' scratch document that is the first step in generating
% the index for the Vegetariano book.
%
% This scratch document generates a file that includes all the 'temporary workings' of the
% process including snippets all the sections, recipe titles and ingredients. By itself it
% is not really that useful, but can be used to track down 'bugs' and review the index that
% is generated.
%
% Really it is a stepping stone to then generate a standalone index, and then a booklet form
% index for printing.
%
% What this document does contain is all the special commands we use to construct the index
% from notes of the sections, titles and ingredients of the recipes.

\documentclass[12pt,a4paper]{book}

%\title{Vegetariano Index}
%\author{Graham Whaley}

% %define our default ingredients list layout
% % Note the nasty hbadness stuff.  As we are cutting the ingredients to two columns,
% % we get some underfull box warnings in 12pt, so this stops them appearing.
% \def\beginingredientslist{\begin{multicols}{2} \hbadness=10000 \begin{itemize}}
% \def\endingredientslist{\end{itemize} \hbadness=1000 \end{multicols}}

% Later we may want to adjust this to be 4 columns of tiny font - but whilst deveoping it is
% easier to view as 2 columns of scriptsize
\usepackage[makeindex]{imakeidx}
\indexsetup{othercode=\scriptsize}
\makeindex[columns=2]

% We spit out the actual arguments as text as well as adding them to the index as then in the 'raw'
% document we get actual pages-per-recipe. If we did not then \newpage would effectively get ignored
% and everything would end up on the first, blank, page. We can probably do this 'better' if we force
% a newpage in the actual body of the text, but it's sort of useful to have the ingredients and stuff
% dumped out in the raw text file to aid debug - and it's easy. This will only become a problem if
% an ingredient list overflows a pagefull, which I think is doubtful (and if it does, we can probably
% fix that later by squishing into a paragraph, or forcing page numbers if we have to. Page numbers
% buried in the body text might be useful later anyway

% Use special commands as variables to carry around the current category and recipe names so we can
% use them with ! separators in the indexing of recipes and ingredients.
\newcommand{\Vcurrentcategory}{xxx}
\def\category #1{\hfill \break #1: \index{#1} \renewcommand{\Vcurrentcategory}{#1}}

\newcommand{\Vcurrentrecipe}{xxx}
\def\recipe #1{\hfill \break #1: \index{#1} \index{\Vcurrentcategory!#1} \renewcommand{\Vcurrentrecipe}{#1}}

% Try to cater for multi-page recipes (of which there are so far few...)
% This works in that the main recipe title and section references cover multi-pages, but the
% ingredients still only show the first page. It seems not quite trivial to fix that. I tried
% putting the recipestop() before the ingredients, but that still didn't seem to do it. It's not
% a big problem though.
\def\recipestart #1{\hfill \break #1: \index{#1|(} \index{\Vcurrentcategory!#1|(} \renewcommand{\Vcurrentrecipe}{#1}}
\def\recipestop #1{ \index{#1|)} \index{\Vcurrentcategory!#1|)} }

% I find it slightly odd that the 'see' cross references don't carry page numbers, but it seems that
% is some sort of 'standard' in indexes, and it is not quite trivial to fix with the default \see
% operation. We'll leave it as be for now.
% Ingredient refs don't carry a photo bold ref either
\def\italianrecipe #1{\hfill \break #1: \index{#1@\textsl{#1}|see {\Vcurrentrecipe}} }

% 'altsort' lets us set the sort place in the index for oddly named recipes, such as the pasta
% beggining with "'O Sicchio", which otherwise ends up at the front of the index
% Order of arguments: {Real Name}{Sort Name}
\def\italianrecipealtsort #1#2{\hfill \break #1: \index{#2@\textsl{#1}|see {\Vcurrentrecipe}} }
\def\photo #1{\hfill \break #1: \index{#1|textbf} \index{\Vcurrentcategory!#1|textbf} }

% Add a method to allow references to the current recipe - a bit like 'italianrecipe'. Very useful
% when a recipe contains a hidden sub-recipe (such as how to make a sauce etc.). Do this as a def as
% not only is it 'cleaner', but we need it to expand so we can process the index file later to make
% the single index - outside of this main body text.
% Warning - seems if we hard-code a \Vcurrentrecipe reference outside of a def then it will get
% post-evaluated later, and pick up the wrong recipe name!
\def\refcurrentrecipe #1{\hfill \break #1: \index{#1@\textsl{#1}|see {\Vcurrentrecipe}} }
% Order of arguments: {Real Name}{Sort Name}
\def\refcurrentrecipealtsort #1#2{\hfill \break #1: \index{#2@\textsl{#1}|see {\Vcurrentrecipe}} }

% Later we may make use of the region, although the book itself does contain a region based index.
% For now, insert the data into the file, but do nothing with it.
\def\region #1{}

\def\ingredient #1{#1 \index{#1!\Vcurrentrecipe}}

% ingredient_redundant is used when an ingredient is correct, but really just adds duplication to
% the index. For instance, if we have 'Apricot Dumplings' as the only recipe that uses 'Apricots'
% as an ingredient, then the index looks like:
%
%  Apricot Dumplings, 384
%  Apricots
%    Apricot Dumplings, 384
%
% If we reduce that to the single direct recipe index entry (by dropping the Apricots as an ingredient)
% then we lose nothing in the index (the recipe is still directly findable by looking for 'Apricots',
% but we save 2 lines and make the index cleaner.
%
% This is mostly used when a recipe name starts with the name of an ingredient, and that recipe is
% the only one to use that ingredient (or, possibly all recipes using that ingredient start with
% the name of that ingredient). It may find other uses though. It is kept as a 'live' command so that
% if somebody wishes to re-insert these lines they merely have to change the below command, and not
% do a global search and replace.
\def\ingredientredundant #1{#1}

\def\vegan {\index{Vegan!\Vcurrentrecipe}}

\begin{document}

This is the first stage document used to build the index for the book 'Vegetariano'.


%%%%%%%%%%%%%%%%%%%%%%%%%%%%% START OF SEEALSO %%%%%%%%%%%%%%%%%%%%%%%%%%%%%%%%%%%%%


\def\seealso#1#2{\index{#1@\textsl{#1}|see {#2}} }

% List out pseudonyms...

% See {From} -> {To}

\seealso{Almonds}{Nuts, Almonds}

\seealso{Beets}{Beetroot}
\seealso{Bell Peppers}{Peppers, Bell}
\seealso{Broad Beans}{Beans, Fava}

\seealso{Calabrese}{Broccoli}
\seealso{Cavalo Nero}{Kale}
\seealso{Cep}{Mushrooms, Porcini}
\seealso{Chanterelle}{Mushrooms, Chanterelle}
\seealso{Chickpeas}{Beans, Chickpeas}
\seealso{Chilli}{Peppers, Chilli}
\seealso{Courgette}{Zucchini}

\seealso{Eggplant}{Aubergines}
\seealso{Emmer}{Farro}
\seealso{Escarole}{Endive}

\seealso{Fava Beans}{Beans, Fava}
\seealso{French Beans}{Beans, Green}

\seealso{Gianduja}{Gianduia}
\seealso{Green Beans}{Beans, Green}

\seealso{Hazelnuts}{Nuts, Hazelnuts}

\seealso{Melanzane}{Aubergines}

\seealso{Nutella}{Gianduia}

\seealso{Parmesan}{Cheese, Parmesan}

\seealso{Walnuts}{Nuts, Walnuts}

\seealso{Zucchini Blossoms}{Squash Blossoms}

%%%%%%%%%%%%%%%%%%%%%%%%%%%%% END OF SALADS %%%%%%%%%%%%%%%%%%%%%%%%%%%%%%%%%%%%%

%%%%%%%%%%%%%%%%%%%%%%%%%%%%% START OF SOUPS %%%%%%%%%%%%%%%%%%%%%%%%%%%%%%%%%%%%%

% First recipe is on page 18

\setcounter{page}{18} %18 %%%%%%%%%%%%%%%%%%%%%%%%%%%%%%%%%%%%%%%%%%%%%%%%%%%%%%%%%%%%%%%%%%

\category{Soups}
\recipe{Spring Vegetable Soup}
\italianrecipe{Bazzoffia}
\region{Lazio}
\ingredient{Artichokes, Globe}
\ingredient{Chard}
\ingredient{Beans, Fava}

\recipe{Tiny Dumplings In Milk}
\italianrecipe{Menestra de fregoloti}
\region{Trentino}
% No standout ingredients!

\newpage	%19 %%%%%%%%%%%%%%%%%%%%%%%%%%%%%%%%%%%%%%%%%%%%%%%%%%%%%%%%%%%%%%%%%%

\recipe{Vegetable Stew With Semolina Pasta}
\italianrecipe{Minestrone}
\region{Calabria}
\ingredient{Chard}
\ingredient{Zucchini}
\ingredient{Beans, Green}

\newpage	%20 %%%%%%%%%%%%%%%%%%%%%%%%%%%%%%%%%%%%%%%%%%%%%%%%%%%%%%%%%%%%%%%%%%

\recipe{Vegetable Soup With Pesto}
\italianrecipe{Minestrone alla Genovese}
\region{Liguria}
\ingredient{Borlotti Beans}
\ingredient{Pesto}

\newpage	%21 %%%%%%%%%%%%%%%%%%%%%%%%%%%%%%%%%%%%%%%%%%%%%%%%%%%%%%%%%%%%%%%%%%
\photo{Minestrone alla Genovese}

\newpage	%22 %%%%%%%%%%%%%%%%%%%%%%%%%%%%%%%%%%%%%%%%%%%%%%%%%%%%%%%%%%%%%%%%%%

\recipe{Stewed Fava Beans, Peas, Artichokes and Lettuce}
\italianrecipe{Fittedda}
\region{Sicilia}
\vegan
\ingredient{Beans, Fava}
\ingredient{Peas}
\ingredient{Artichokes, Globe}
\ingredient{Lettuce}

\recipe{Enriched Milk Soup}
\italianrecipe{Mosa}
\region{Trentino}
\ingredient{Milk}
\ingredient{Cornmeal}

\newpage	%23 %%%%%%%%%%%%%%%%%%%%%%%%%%%%%%%%%%%%%%%%%%%%%%%%%%%%%%%%%%%%%%%%%%

\recipe{Enriched Vegetable Soup}
\italianrecipe{Spatatata}
\region{Lazio}
% No primary key ingredients

\recipe{Leek and Onion Soup with Rye Bread}
\italianrecipe{Skil\a`a}
\region{Valle d'Aosta}
\ingredient{Leeks}
\ingredient{Onions}
\ingredient{Rye Bread}

\newpage	%24 %%%%%%%%%%%%%%%%%%%%%%%%%%%%%%%%%%%%%%%%%%%%%%%%%%%%%%%%%%%%%%%%%%

\recipe{Vegetable Stew}
\italianrecipe{Ciambottella}
\vegan
\region{Campania}
\ingredient{Zucchini}
\ingredient{Aubergines}

\recipe{Potato Puree with Kale}
\italianrecipe{Prebugiun di Patate e Cavolo Nero}
\vegan
\region{Liguria}
\ingredient{Potatoes}
% \ingredient{Cavolo Nero} % ->Kale in seealso
\ingredient{Kale}

\newpage	%25 %%%%%%%%%%%%%%%%%%%%%%%%%%%%%%%%%%%%%%%%%%%%%%%%%%%%%%%%%%%%%%%%%%

\recipe{Bread Soup with Tomatoes}
\italianrecipe{Acquacotta Maremmana}
\region{Toscana}
\ingredient{Bread}
\ingredient{Tomatoes}

\newpage	%26 %%%%%%%%%%%%%%%%%%%%%%%%%%%%%%%%%%%%%%%%%%%%%%%%%%%%%%%%%%%%%%%%%%

\recipe{Cold Cucumber Soup}
\italianrecipe{Zuppa di Cetrioli}
\region{Alto Adige}
\ingredient{Cucumber}
\ingredient{Tomatoes}

\newpage	%27 %%%%%%%%%%%%%%%%%%%%%%%%%%%%%%%%%%%%%%%%%%%%%%%%%%%%%%%%%%%%%%%%%%

\photo{Cold Cucumber Soup}

\newpage	%28 %%%%%%%%%%%%%%%%%%%%%%%%%%%%%%%%%%%%%%%%%%%%%%%%%%%%%%%%%%%%%%%%%%

\recipe{Tomato Soup}
\italianrecipe{Pappa al Pomodoro}
\vegan
\region{Toscana}
\ingredient{Tomatoes}
\ingredient{Bread}

\newpage	%29 %%%%%%%%%%%%%%%%%%%%%%%%%%%%%%%%%%%%%%%%%%%%%%%%%%%%%%%%%%%%%%%%%%

\photo{Tomato Soup}

\newpage	%30 %%%%%%%%%%%%%%%%%%%%%%%%%%%%%%%%%%%%%%%%%%%%%%%%%%%%%%%%%%%%%%%%%%

\recipe{Bell Pepper Soup}
\italianrecipe{Crema di Peperoni}
\region{Toscana}
\ingredient{Peppers, Bell}

\recipe{Cheese Curd Soup}
\italianrecipe{Minestra di Vrughe}
\region{Sardegna}
\ingredient{Potatoes}
\ingredient{Cheese}
\ingredient{Curds}

\newpage	%31 %%%%%%%%%%%%%%%%%%%%%%%%%%%%%%%%%%%%%%%%%%%%%%%%%%%%%%%%%%%%%%%%%%

% Might want to reduce the length of this title ?
\recipe{Puree of Caprauna Turnips with popped Amaranth}
\italianrecipe{Crema di Rape di Caprauna con Amaranto}
\region{Trentino}
\ingredient{Amaranth}
\ingredient{Turnips}

\newpage	%32 %%%%%%%%%%%%%%%%%%%%%%%%%%%%%%%%%%%%%%%%%%%%%%%%%%%%%%%%%%%%%%%%%%

\recipe{Mushroom soup}
\italianrecipe{Zuppa di Funghi}
\region{Trentino}
\ingredient{Mushrooms, Porchini}
\ingredient{Mushrooms, Chanterelle}

\newpage	%33 %%%%%%%%%%%%%%%%%%%%%%%%%%%%%%%%%%%%%%%%%%%%%%%%%%%%%%%%%%%%%%%%%%

\photo{Mushroom soup}

\newpage	%34 %%%%%%%%%%%%%%%%%%%%%%%%%%%%%%%%%%%%%%%%%%%%%%%%%%%%%%%%%%%%%%%%%%

\recipe{Rice and Turnip Soup}
\italianrecipe{Minestra di riso e rape}
\region{Valle d'Aosta}
\ingredient{Rice}
\ingredient{Turnips}

\recipe{Rice and Egg Noodle Soup}
\italianrecipe{Minestra di risi e tajadele}
\region{Veneto}
\ingredient{Rice}
\ingredient{Tagliatelle}

\newpage	%35 %%%%%%%%%%%%%%%%%%%%%%%%%%%%%%%%%%%%%%%%%%%%%%%%%%%%%%%%%%%%%%%%%%

\photo{Rice and Egg Noodle Soup}

\newpage	%36 %%%%%%%%%%%%%%%%%%%%%%%%%%%%%%%%%%%%%%%%%%%%%%%%%%%%%%%%%%%%%%%%%%

\recipe{Bread Soup for Ribollita}
\italianrecipe{Minestra di Pane per la Ribollita}
\vegan
\region{Toscana}
% The Ribollita recipe is rolled into the end of the soup recipe
%\index{Ribbolita|see {\Vcurrentrecipe}}
\refcurrentrecipe{Ribbolita}
\ingredient{Cannellini Beans}
\ingredient{Bread}
% There are many ingredients in this, but the beans, Kale and Cabbage are key
\ingredient{Kale}
\ingredient{Cabbage}

\newpage	%37 %%%%%%%%%%%%%%%%%%%%%%%%%%%%%%%%%%%%%%%%%%%%%%%%%%%%%%%%%%%%%%%%%%

\recipe{Fontina Cheese Soup}
\italianrecipe{Zuppa alla Fontina}
\region{Valle d'Aosta}
\ingredient{Cheese, Fontina}
\ingredient{Bread}

\newpage	%38 %%%%%%%%%%%%%%%%%%%%%%%%%%%%%%%%%%%%%%%%%%%%%%%%%%%%%%%%%%%%%%%%%%

\recipe{Onion Soup with Melted Cheese}
\italianrecipe{Zuppa di Cipolle}
\region{Valle d'Aosta}
\ingredient{Onions}
\ingredient{Cheese, Gruy\'{e}re}
\ingredient{Cheese, Parmesan}

\newpage	%39 %%%%%%%%%%%%%%%%%%%%%%%%%%%%%%%%%%%%%%%%%%%%%%%%%%%%%%%%%%%%%%%%%%

\recipe{Potato and Milk Soup with Torn Pasta}
\italianrecipe{Spisucul\"{o}ch}
\region{Trentino}
\ingredient{Potatoes}
\ingredient{Cheese, Spressa}
% \index{Cheese, Spressa}	% As an individual entry this is confusing

\newpage	%40 %%%%%%%%%%%%%%%%%%%%%%%%%%%%%%%%%%%%%%%%%%%%%%%%%%%%%%%%%%%%%%%%%%

\recipe{Bean and Cabbage Soup}
\italianrecipe{Cucina}
\vegan
\region{Toscana}
\ingredient{Borlotti Beans}
\ingredient{Cabbage}
\ingredient{Cabbage, Savoy}
\ingredient{Wild Greens}

\newpage	%41 %%%%%%%%%%%%%%%%%%%%%%%%%%%%%%%%%%%%%%%%%%%%%%%%%%%%%%%%%%%%%%%%%%

\recipe{Wild Greens, Beans and Cheese Curd Soup}
\italianrecipe{Erbuzza}
\region{Sardegna}
\ingredient{Beans}
\ingredient{Wild Greens}
\ingredient{Cheese}
\ingredient{Curds}

\newpage	%42 %%%%%%%%%%%%%%%%%%%%%%%%%%%%%%%%%%%%%%%%%%%%%%%%%%%%%%%%%%%%%%%%%%

% The actual recipe says 'Chick Pea', but that then looks odd in the index, and all
% the ingredients say 'Chickpea'. I looked it up - you can also spell it 'Chick-Pea'.
% We'll unify on on un-spaced un-hyphenated word...
\recipe{Chickpea and Chestnut Puree}
\italianrecipe{Ceci e Castagne}
\region{Abruzzo}
\ingredient{Beans, Chickpeas}
\ingredient{Chestnuts}

\newpage	%43 %%%%%%%%%%%%%%%%%%%%%%%%%%%%%%%%%%%%%%%%%%%%%%%%%%%%%%%%%%%%%%%%%%

\photo{Chickpea and Chestnut Puree}

\newpage	%44 %%%%%%%%%%%%%%%%%%%%%%%%%%%%%%%%%%%%%%%%%%%%%%%%%%%%%%%%%%%%%%%%%%

\recipe{Fava, Cabbage, Fennel and Borage Soup}
\italianrecipe{Caulada Vegetariana}
\vegan
\region{Sardegna}
\ingredient{Beans, Fava}
\ingredient{Cabbage}
\ingredient{Fennel, Bulb}
\ingredient{Cabbage}
\ingredient{Cabbage, Savoy}

\newpage	%45 %%%%%%%%%%%%%%%%%%%%%%%%%%%%%%%%%%%%%%%%%%%%%%%%%%%%%%%%%%%%%%%%%%

\recipe{Lentil Soup}
\italianrecipe{Zuppa di Lenticchie}
\vegan
\region{Sicilia}
% Do we need both Lentils and Lentils Ustica? Probably not
% As the recipe starts with the word 'Lentil', let's just drop the plain Lentil ingredient
%\ingredient{Lentils}
\ingredient{Lentils, Ustica}

%%%%%%%%%%%%%%%%%%%%%%%%%%%%% END OF SOUPS %%%%%%%%%%%%%%%%%%%%%%%%%%%%%%%%%%%%%

%%%%%%%%%%%%%%%%%%%%%%%%%%%%% START OF SALADS %%%%%%%%%%%%%%%%%%%%%%%%%%%%%%%%%%%%%

% Technically, 'Salads and Composed Vegetable Dishes'

% First recipe is on page 48 - set to 47, and then bump to new page

\setcounter{page}{47} %47 %%%%%%%%%%%%%%%%%%%%%%%%%%%%%%%%%%%%%%%%%%%%%%%%%%%%%%%%%%%%%%%%%%

\newpage	%48 %%%%%%%%%%%%%%%%%%%%%%%%%%%%%%%%%%%%%%%%%%%%%%%%%%%%%%%%%%%%%%%%%%

\category{Salads}
\recipe{Apple and Celeriac Salad}
\italianrecipe{Insalata di Sedano Rapa e Mele}
\region{Valle d'Aosta}
\ingredient{Celeriac}
\ingredient{Apples}
\ingredient{Mayonnaise}	% Technically, the recipe tells how to make Mayo
% Also add a ref to Mayo in the index. That makes an 'extra' ref, but it helps make it clear
% that this is a recipe, not just an ingredient
%\index{Mayonnaise|see {\Vcurrentrecipe}}
\refcurrentrecipe{Mayonnaise}

\category{Salads}
\recipe{Beet Salad with Yoghurt Dressing}
\italianrecipe{Insalata di Barbabietole}
\region{Friuli Venezia Giulia}
\ingredient{Yoghurt}
\ingredient{Beetroot}
\ingredient{Mesclun}	% OK, it's just a topping, but it's quite a unique word to index :-)

\newpage	%49 %%%%%%%%%%%%%%%%%%%%%%%%%%%%%%%%%%%%%%%%%%%%%%%%%%%%%%%%%%%%%%%%%%

\recipe{Mixed Vegetable Salad with Mayonnaise}
\italianrecipe{Insalata Russa}
\refcurrentrecipe{Olivier Salad}	% Apparently another name for it
\refcurrentrecipe{Russian Salad}	% Apparently another name for it
\region{Piemonte}
% There are many variations - it is quite hard to determine what the key elements are
\ingredient{Potatoes}
\ingredient{Peas}
\ingredient{Carrots}
\ingredient{Zucchini}	%they are effectively made into pickles - pickles being a regular
% ingredient in this recipe...
\refcurrentrecipe{Pickles}

\ingredient{Mayonnaise}
% How bizzare - the recipe has a tin of Tuna in the ingredients list, but it is not
% used in the actual recipe!! Most odd!! And most un-vegetarian ;-)

\newpage	%50 %%%%%%%%%%%%%%%%%%%%%%%%%%%%%%%%%%%%%%%%%%%%%%%%%%%%%%%%%%%%%%%%%%

\recipe{Sauerkraut Salad}
\italianrecipe{Insalata di Cauti}
\vegan
\region{Trentino}
\ingredient{Sauerkraut}
\ingredient{Apples}

\recipe{Green Cabbage Salad}
\italianrecipe{Insalata di Vacolo Cappuccio}
\region{Trentino}
\ingredient{Cabbage}
% Tricky one here - this shows up strange in the index as you get croutons referencing this
% single recipe twice - but, it is technically correct, as one is an ingredient reference and
% the other an italicized recipe reference - so, we leave as is right now.
\ingredient{Croutons}
\refcurrentrecipe{Croutons}

\newpage	%51 %%%%%%%%%%%%%%%%%%%%%%%%%%%%%%%%%%%%%%%%%%%%%%%%%%%%%%%%%%%%%%%%%%

\recipe{Salad with Fromadzo Cheese}
\italianrecipe{Insalata di Verdure e Fromadzo}
\region{Valle d'Aosta}
\ingredient{Radish}
\ingredient{Lettuce}
\ingredient{Cheese, Fromadzo}
\ingredient{Cheese}

\recipe{Pea Pod Salad}
\italianrecipe{Insalata di Bucce di Piselli}
\vegan
\region{Piemonte}
\ingredientredundant{Pea Pods}
\ingredient{Lettuce}
\ingredient{Cheese, Fromadzo}

\newpage	%52 %%%%%%%%%%%%%%%%%%%%%%%%%%%%%%%%%%%%%%%%%%%%%%%%%%%%%%%%%%%%%%%%%%

\recipe{Red Cabbage Salad}
\italianrecipe{Insalata di Cavolo Rosso}
\region{Friuli Venezia Giulia}
\ingredient{Cabbage, Red}
\ingredient{Cabbage}
\ingredient{Oranges}
\ingredient{Apples}

\recipe{Salad of Cauliflower, Capers and Olives}
\italianrecipe{Insalata di Cavolfiore}
\region{Campania}
\ingredient{Cauliflower}
\ingredient{Capers}
\ingredient{Olives}

\newpage	%53 %%%%%%%%%%%%%%%%%%%%%%%%%%%%%%%%%%%%%%%%%%%%%%%%%%%%%%%%%%%%%%%%%%

\photo{Red Cabbage Salad}

\newpage	%54 %%%%%%%%%%%%%%%%%%%%%%%%%%%%%%%%%%%%%%%%%%%%%%%%%%%%%%%%%%%%%%%%%%

\recipe{Salad of Dandelion, Apple and Chives}
\italianrecipe{Tarassaco, mele ed erba Cipollina}
\region{Emilia-Romagna}
\vegan
\ingredient{Dandelion}
\ingredient{Chives}
\ingredient{Apples}

\recipe{Sliced Orange Salad}
\italianrecipe{Insalata di Arance}
\region{Sicilia}
\vegan
\ingredient{Oranges}

\newpage	%55 %%%%%%%%%%%%%%%%%%%%%%%%%%%%%%%%%%%%%%%%%%%%%%%%%%%%%%%%%%%%%%%%%%

\recipe{Raw Artichoke Salad}
\italianrecipe{Insalata di Castraure}
\region{Veneto}
\vegan
\ingredient{Artichokes, Globe}

\newpage	%56 %%%%%%%%%%%%%%%%%%%%%%%%%%%%%%%%%%%%%%%%%%%%%%%%%%%%%%%%%%%%%%%%%%

\recipe{Fava and Celery Salad with Wild Oregano}
\italianrecipe{Insalata di Cottoia}
\region{Sicilia}
\vegan
\ingredient{Beans, Fava}
\ingredient{Beans}
\ingredient{Celery}
\ingredient{Oregano}

\recipe{Zucchini and Pine Nut Salad}
\italianrecipe{Insalata di Zucchine e Pinoli}
\region{Emilia-Romagna}
\ingredient{Zucchini}
\ingredient{Pine Nuts}

\newpage	%57 %%%%%%%%%%%%%%%%%%%%%%%%%%%%%%%%%%%%%%%%%%%%%%%%%%%%%%%%%%%%%%%%%%

\recipe{Wild Greens Salad}
\italianrecipe{Insalata di Primavera}
\region{Piemonte}
\vegan
\ingredient{Lettuce}
\ingredient{Wild Greens}

\recipe{Couscous Salad}
\italianrecipe{Insalata di Cuscus}
\region{Sardegna}
\ingredientredundant{Couscous}
\ingredient{Zucchini}
\ingredient{Carrots}
\ingredient{Tomatoes}

\newpage	%58 %%%%%%%%%%%%%%%%%%%%%%%%%%%%%%%%%%%%%%%%%%%%%%%%%%%%%%%%%%%%%%%%%%

\recipe{Vegetable Salad with Farro}
\italianrecipe{Insalata di Farro e Verdure}
\region{Abruzzo}
\vegan
\ingredient{Farro}
% Pretty much all other ingredients look like they can be substituted - the only key
% ingredient is the Farro

\newpage	%59 %%%%%%%%%%%%%%%%%%%%%%%%%%%%%%%%%%%%%%%%%%%%%%%%%%%%%%%%%%%%%%%%%%

\recipe{Beans, Squash and Potatoes with Garlicky Dressing}
\italianrecipe{Patate Voiani e Cucuzzedja}
\region{Calabria}
\vegan
\ingredient{Cucuzzedia}
\ingredient{Squash}
% I can't find 'Voiani' Beans on the internet - nearest I can find is they might be 'Roman' beans,
% which, are our UK Runner beans??
\ingredient{Beans, Flat}
\ingredient{Runner Beans}
\ingredient{Potatoes}

\newpage	%60 %%%%%%%%%%%%%%%%%%%%%%%%%%%%%%%%%%%%%%%%%%%%%%%%%%%%%%%%%%%%%%%%%%

\recipe{Bread Salad with Ripe Tomatoes}
\italianrecipe{Panzanella}
% The recipe also includes another hidden recipe
% We already ref Courgette over to Zucchini
%\refcurrentrecipe{Stuffed Courgette Flowers}
\refcurrentrecipe{Squash Blossoms, Stuffed}
\italianrecipe{Fiori di Zucca Fritti}
\region{Toscana}
\vegan
% Tricky one this - the Squash Blossoms are actually for the 'hidden' recipe, not
% the main recipe - but, if we miss them out, it will make the hidden recipe harder to find.
\ingredient{Squash Blossoms}
\ingredient{Bread}
\ingredient{Tomatoes}

\recipe{Sweet and Sour Beets}
\italianrecipe{Barbabietole in Agrodolce}
\region{Trentino}
\vegan
\ingredient{Beetroot}

\newpage	%61 %%%%%%%%%%%%%%%%%%%%%%%%%%%%%%%%%%%%%%%%%%%%%%%%%%%%%%%%%%%%%%%%%%

\photo{Bread Salad with Ripe Tomatoes}

\newpage	%62 %%%%%%%%%%%%%%%%%%%%%%%%%%%%%%%%%%%%%%%%%%%%%%%%%%%%%%%%%%%%%%%%%%

\recipe{Sweet and Sour Cipollini Onions}
\italianrecipe{Cipolline in Agrodolce}
\region{Veneto}
\vegan
\ingredient{Onions}
\ingredient{Raisins}

\recipe{Green Peppers Preserved in Vinegar}
\italianrecipe{Tighe Sott'aceto}
\region{Lombardia}
\vegan
\ingredient{Peppers, Green}

\newpage	%63 %%%%%%%%%%%%%%%%%%%%%%%%%%%%%%%%%%%%%%%%%%%%%%%%%%%%%%%%%%%%%%%%%%

\photo{Sweet and Sour Cipollini Onions}

\newpage	%64 %%%%%%%%%%%%%%%%%%%%%%%%%%%%%%%%%%%%%%%%%%%%%%%%%%%%%%%%%%%%%%%%%%

\recipe{Preserved Sweet and Sour Vegetables}
\italianrecipe{Verdure Agrodolci}
\region{Veneto}
\vegan
\ingredient{Tomatoes}

\newpage	%65 %%%%%%%%%%%%%%%%%%%%%%%%%%%%%%%%%%%%%%%%%%%%%%%%%%%%%%%%%%%%%%%%%%

\recipe{Red Giardiniera with Tomato Sauce}
\italianrecipe{Giardiniera Rossa}
\region{Piemonte}
\vegan
\ingredient{Tomatoes}
% Selected list taken from Wikipedia reference :-)
\ingredient{Peppers, Bell}
\ingredient{Celery}
\ingredient{Cauliflower}
\ingredient{Carrots}
\ingredient{Zucchini}

\newpage	%66 %%%%%%%%%%%%%%%%%%%%%%%%%%%%%%%%%%%%%%%%%%%%%%%%%%%%%%%%%%%%%%%%%%

\recipe{Bell Peppers Preserved in Sweet and Sour Sauce}
\italianrecipe{Peperoni in Agrodolce}
\region{Piemonte}
\vegan
\ingredient{Peppers, Bell}

\newpage	%67 %%%%%%%%%%%%%%%%%%%%%%%%%%%%%%%%%%%%%%%%%%%%%%%%%%%%%%%%%%%%%%%%%%

\photo{Bell Peppers Preserved in Sweet and Sour Sauce}

\newpage	%68 %%%%%%%%%%%%%%%%%%%%%%%%%%%%%%%%%%%%%%%%%%%%%%%%%%%%%%%%%%%%%%%%%%

\recipe{Hot Peppers Preserved in Oil}
\italianrecipe{Peproncini sott'olio}
\region{Abruzzo}
\vegan
\ingredient{Peppers, Chilli}

\recipe{Eggplant Rolls in Olive Oil}
\italianrecipe{Involtini di Melanzane sott'olio}
\region{Calabria}
\vegan
\ingredient{Aubergines}
\ingredient{Capers}

\newpage	%69 %%%%%%%%%%%%%%%%%%%%%%%%%%%%%%%%%%%%%%%%%%%%%%%%%%%%%%%%%%%%%%%%%%

\recipe{Garlic Scapes in Olive Oil}
\italianrecipe{Zolle sott'olio}
\region{Abruzzo}
\vegan
\ingredient{Garlic, Stems}

\recipe{Baby Chicory Preserved in Oil}
\italianrecipe{Cicoriette sott'olio}
\region{Basilicata}
\vegan
\ingredient{Chicory}

\newpage	%70 %%%%%%%%%%%%%%%%%%%%%%%%%%%%%%%%%%%%%%%%%%%%%%%%%%%%%%%%%%%%%%%%%%

\recipe{Olives with Mint and Garlic}
\italianrecipe{Alivi Cunzati}
\region{Sicilia}
\vegan
\ingredient{Olives, Fresh}
\ingredient{Garlic}
\ingredient{Mint}
\ingredient{Lemons}

\newpage	%71 %%%%%%%%%%%%%%%%%%%%%%%%%%%%%%%%%%%%%%%%%%%%%%%%%%%%%%%%%%%%%%%%%%

\photo{Olives with Mint and Garlic}

\newpage	%72 %%%%%%%%%%%%%%%%%%%%%%%%%%%%%%%%%%%%%%%%%%%%%%%%%%%%%%%%%%%%%%%%%%

\recipe{Chiodino and Straw Mushrooms Preserved in Oil}
\italianrecipe{Chodini e Funghi di Muschio sott'olio}
\region{Basilicata}
\vegan
\ingredient{Mushrooms}

\recipe{Vegetables dipped in Olive Oil}
\italianrecipe{Bagn\'{e} 'nt L'Euli}
\region{Piedmonte}
\vegan
\ingredient{Olive Oil}	% Because it is the only real key element!
% Err, no specific vegetables - random selection!

\newpage	%73 %%%%%%%%%%%%%%%%%%%%%%%%%%%%%%%%%%%%%%%%%%%%%%%%%%%%%%%%%%%%%%%%%%

\photo{Chiodino and Straw Mushrooms Preserved in Oil}

\newpage	%74 %%%%%%%%%%%%%%%%%%%%%%%%%%%%%%%%%%%%%%%%%%%%%%%%%%%%%%%%%%%%%%%%%%

\recipe{Marinated Roasted Eggplant}
\italianrecipe{Melanzane in Saor}
\region{Veneto}
\vegan
\ingredient{Aubergines}

\newpage	%75 %%%%%%%%%%%%%%%%%%%%%%%%%%%%%%%%%%%%%%%%%%%%%%%%%%%%%%%%%%%%%%%%%%

\recipe{Celery with Onion Spread}
\italianrecipe{Coste di Sedano con Cipollata}
\region{Toscana}
\ingredient{Peppers, Bell}
\ingredient{Onions, Spring}
\ingredient{Cheese, Robiola}

\recipe{Vegetables in Balsamic Vinegar}
\italianrecipe{Verdurine all'aceto Balsamico}
\region{Emilia-Romagna}
\vegan
\ingredient{Vinegar, Balsamic}
\ingredient{Carrots}
\ingredient{Fennel, Bulb}
\ingredient{Broccoli}

\newpage	%76 %%%%%%%%%%%%%%%%%%%%%%%%%%%%%%%%%%%%%%%%%%%%%%%%%%%%%%%%%%%%%%%%%%

\recipe{Zucchini and Tomatoes stuffed with Ricotta}
\italianrecipe{Zucchine e Pomodori con Ricotta}
\region{Lazio}
\ingredient{Zucchini}
\ingredient{Tomatoes}
\ingredient{Cheese, Ricotta}

\newpage	%77 %%%%%%%%%%%%%%%%%%%%%%%%%%%%%%%%%%%%%%%%%%%%%%%%%%%%%%%%%%%%%%%%%%

\photo{Zucchini and Tomatoes stuffed with Ricotta}

\newpage	%78 %%%%%%%%%%%%%%%%%%%%%%%%%%%%%%%%%%%%%%%%%%%%%%%%%%%%%%%%%%%%%%%%%%

\recipe{Caciofiore Cheese with Viterbo Carrots}
\italianrecipe{Caciofiore con Catote di Viterbo}
\region{Lazio}
\ingredient{Carrots}
\ingredient{Cheese, Caciofiore}

\newpage	%79 %%%%%%%%%%%%%%%%%%%%%%%%%%%%%%%%%%%%%%%%%%%%%%%%%%%%%%%%%%%%%%%%%%

\null	% Force something before the newpage, or we drop the empty page and get one
	% page out of step...
%\photo{???}	% There is a photo on this page, but I'm not sure what of??

\newpage	%80 %%%%%%%%%%%%%%%%%%%%%%%%%%%%%%%%%%%%%%%%%%%%%%%%%%%%%%%%%%%%%%%%%%

\recipe{Tart Cardoons}
\italianrecipe{Cardi Acidi}
\vegan
\region{Lazio}
\ingredient{Cardoons}

\newpage	%81 %%%%%%%%%%%%%%%%%%%%%%%%%%%%%%%%%%%%%%%%%%%%%%%%%%%%%%%%%%%%%%%%%%

\recipe{Montasio Cheese with Fruit Compote}
\italianrecipe{Montasio con Composta di Frutta}
\region{Friuli Venezia Giulia}
\ingredient{Cheese, Montasio}
% There is a hidden fruit compote recipe here
\refcurrentrecipe{Fruit Compote}

\newpage	%82 %%%%%%%%%%%%%%%%%%%%%%%%%%%%%%%%%%%%%%%%%%%%%%%%%%%%%%%%%%%%%%%%%%

\recipe{Marinated Radicchio}
\italianrecipe{Radicchio Marinato}
\vegan
\region{Veneto}
\ingredient{Radicchio}

\recipe{Potatoes and Green Beans in Pesto}
\italianrecipe{Patate e Fagiolini con Pesto}
\region{Liguria}
\ingredient{Potatoes}
\ingredient{Beans, Green}
\ingredient{Pesto}

\newpage	%83 %%%%%%%%%%%%%%%%%%%%%%%%%%%%%%%%%%%%%%%%%%%%%%%%%%%%%%%%%%%%%%%%%%

\photo{Marinated Radicchio}

%%%%%%%%%%%%%%%%%%%%%%%%%%%%% END OF SALADS %%%%%%%%%%%%%%%%%%%%%%%%%%%%%%%%%%%%%

%%%%%%%%%%%%%%%%%%%%%%%%%%%%% START OF CROSTINI %%%%%%%%%%%%%%%%%%%%%%%%%%%%%%%%%%%%%

% Technically, 'Crostini and Toasts'

% First recipe is on page 86 - set to 85, and then bump to new page
\setcounter{page}{85} %85 %%%%%%%%%%%%%%%%%%%%%%%%%%%%%%%%%%%%%%%%%%%%%%%%%%%%%%%%%%%%%%%%%%

\newpage	%86 %%%%%%%%%%%%%%%%%%%%%%%%%%%%%%%%%%%%%%%%%%%%%%%%%%%%%%%%%%%%%%%%%%
\category{Crostini}

\recipe{Salignon Cheese Crostini}
\italianrecipe{Crostini al Salignon}
\region{Valle d'Aosta}
\ingredient{Cheese, Ricotta}

\recipe{Crostini with Shallots}
\italianrecipe{Crostini con Salsa allo Scalogno}
\region{Veneto}
\ingredient{Shallots}
\ingredient{Egg, Yolks}

\newpage	%87 %%%%%%%%%%%%%%%%%%%%%%%%%%%%%%%%%%%%%%%%%%%%%%%%%%%%%%%%%%%%%%%%%%

\recipe{Crostini with Artichokes}
\italianrecipe{Crostini con Gambi di Carciofi}
\region{Emilia-Romagna}
\ingredient{Artichokes, Globe}
\ingredient{Cheese, Raviggiolo}

\recipe{Crostini with Cauliflower}
\italianrecipe{Crostini con Cavolfiori}
\vegan
\region{Marche}
\ingredient{Cauliflower}

\newpage	%88 %%%%%%%%%%%%%%%%%%%%%%%%%%%%%%%%%%%%%%%%%%%%%%%%%%%%%%%%%%%%%%%%%%

\recipe{Truffle Crostini}
\italianrecipe{Crostini con Salsa al Tartufo}
\region{Umbria}
\ingredient{Truffles}

\recipe{Aioli for Crostini}
\italianrecipe{Aj\'{o}li}
\region{Piemonte}
\ingredient{Egg, Yolks}

\newpage	%89 %%%%%%%%%%%%%%%%%%%%%%%%%%%%%%%%%%%%%%%%%%%%%%%%%%%%%%%%%%%%%%%%%%

\recipe{Bread Slices rubbed with Ripe Tomatoes}
\italianrecipe{Acquasalsa}
\vegan
\region{Puglia}
\ingredient{Tomatoes}

\recipe{Semolina Bread with Tomatoes and Sheep's Cheese}
\italianrecipe{Pani Cunzatu}
\region{Sicilia}
\ingredient{Tomatoes}
\ingredient{Cheese, Sheep's}

\newpage	%90 %%%%%%%%%%%%%%%%%%%%%%%%%%%%%%%%%%%%%%%%%%%%%%%%%%%%%%%%%%%%%%%%%%

\recipe{Thin Flatbreads with Herbs and Cheese}
\italianrecipe{Pane Guttiau}
\region{Sardegna}
\ingredient{Cheese, Sheep's}

\recipe{Fried Bread with Peas and Brocolli Rabe}
\italianrecipe{Cecamariti}
\region{Puglia}
\ingredient{Peas}
\ingredient{Broccoli, Rabe}	% Which seems to be not even sprouting broccoli... it is Rabini

\newpage	%91 %%%%%%%%%%%%%%%%%%%%%%%%%%%%%%%%%%%%%%%%%%%%%%%%%%%%%%%%%%%%%%%%%%

\photo{Thin Flatbreads with Herbs and Cheese}

\newpage	%92 %%%%%%%%%%%%%%%%%%%%%%%%%%%%%%%%%%%%%%%%%%%%%%%%%%%%%%%%%%%%%%%%%%

\recipe{Kale Toast}
\italianrecipe{Fetta con Cavolo Nero}
\vegan
\region{Toscana}
\ingredient{Kale}

\recipe{Toasted Garlic Bread}
\italianrecipe{Fettunta}
\vegan
\region{Toscana}
\ingredient{Garlic}

\recipe{Toasted Bread with Chopped Tomatoes}
\italianrecipe{Frega di Pomodoro}
\vegan
\region{Toscana}
\ingredient{Tomatoes}

%%%%%%%%%%%%%%%%%%%%%%%%%%%%% END OF CROSTINI %%%%%%%%%%%%%%%%%%%%%%%%%%%%%%%%%%%%%

%%%%%%%%%%%%%%%%%%%%%%%%%%%%% START OF PASTA %%%%%%%%%%%%%%%%%%%%%%%%%%%%%%%%%%%%%

% First recipe is on page 96 - set to 95, and then bump to new page

\setcounter{page}{95} %95 %%%%%%%%%%%%%%%%%%%%%%%%%%%%%%%%%%%%%%%%%%%%%%%%%%%%%%%%%%%%%%%%%%

\newpage	%96 %%%%%%%%%%%%%%%%%%%%%%%%%%%%%%%%%%%%%%%%%%%%%%%%%%%%%%%%%%%%%%%%%%
\category{Pasta}

\recipe{Spicy Garlic Spaghetti}
\italianrecipe{Spaghetti Aglio Olio e Peproncino}
\vegan
\region{Lazio}
\ingredient{Garlic}
\ingredient{Peppers, Chilli}
% I'm in two minds - I sort of want to add individual pasta types to the index, so you could
% say go look up all the spaghetti recipes - but they are all already under the 'pasta' group,
% so it might get a bit messy. For now, leave them out. We can always add them later if we find
% the index is lacking
%\ingredient{Pasta, Spaghetti}

\newpage	%97 %%%%%%%%%%%%%%%%%%%%%%%%%%%%%%%%%%%%%%%%%%%%%%%%%%%%%%%%%%%%%%%%%%

\recipe{Toasted Hazelnut Spaghetti}
\italianrecipe{Spaghetti con le Nocciole}
\vegan
\region{Campania}
% \ingredient{Hazelnuts} % just see 'nuts'
\ingredient{Nuts, Hazelnuts}

\newpage	%98 %%%%%%%%%%%%%%%%%%%%%%%%%%%%%%%%%%%%%%%%%%%%%%%%%%%%%%%%%%%%%%%%%%

\recipe{Linguine with Capers and Olives}
\italianrecipe{Linguine allo Scammaro}
\region{Campania}
\ingredient{Olives, Green}
\ingredient{Capers}

\newpage	%99 %%%%%%%%%%%%%%%%%%%%%%%%%%%%%%%%%%%%%%%%%%%%%%%%%%%%%%%%%%%%%%%%%%

\recipe{Busiate with Almond and Tomato Pesto}
\italianrecipe{Busiate al Pesto Trapanese}
\region{Sicilia}
\ingredient{Nuts, Almonds}
\ingredient{Olives, Green}
\ingredient{Garlic}	% recipe says must be intensely garlicky
\ingredient{Cheese, Sheep's}
\ingredient{Tomatoes}
\ingredient{Pesto}	% not stricly an ingredient, but good to have an index ref
\refcurrentrecipe{Pesto, Tomato}

\newpage	%100 %%%%%%%%%%%%%%%%%%%%%%%%%%%%%%%%%%%%%%%%%%%%%%%%%%%%%%%%%%%%%%%%%%

\recipe{Trenette with Pesto}
\italianrecipe{Trenette al Pesto}
\region{Liguria}
\refcurrentrecipe{Pesto, Basil}
\refcurrentrecipe{Pesto, Genovese}

\recipe{Vermicelli with Tomato Sauce that Tastes of The Sea}
\italianrecipe{Vermicelli cu' 'o Pesce Fujuto}
\region{Campania}
\vegan
\ingredient{Tomatoes}
\ingredient{Olives, Green}

\newpage	%101 %%%%%%%%%%%%%%%%%%%%%%%%%%%%%%%%%%%%%%%%%%%%%%%%%%%%%%%%%%%%%%%%%%

\recipe{Quick Pasta with Tomatoes}
\italianrecipe{Pasta alla Scarpariello}
\region{Campania}
\ingredient{Cheese, Parmesan}
\ingredient{Tomatoes, Cherry}

\newpage	%102 %%%%%%%%%%%%%%%%%%%%%%%%%%%%%%%%%%%%%%%%%%%%%%%%%%%%%%%%%%%%%%%%%%

\recipe{Pasta with Broccoli Rabe and Cheese}
\italianrecipe{Pasta 'ncaciata con Broccoletti}
\region{Sicilia}
\ingredient{Broccoli, Rabe}
\ingredient{Cheese, Maiorchino}
\ingredient{Cheese, Sheep's}
\ingredient{Cheese, Provolone}

\newpage	%103 %%%%%%%%%%%%%%%%%%%%%%%%%%%%%%%%%%%%%%%%%%%%%%%%%%%%%%%%%%%%%%%%%%

\recipe{Pasta with Tenerumi Leaves}
\italianrecipe{Pasta con i Talli}
\region{Sicilia}
\ingredient{Tenerumi}	% which you may struggle to get - long squash leaves and shoots
\ingredient{Chard}	% an alternative
\ingredient{Spinach}	% an alternative
\ingredient{Tomatoes, Sundried}
\ingredient{Cheese, Sheep's}

\newpage	%104 %%%%%%%%%%%%%%%%%%%%%%%%%%%%%%%%%%%%%%%%%%%%%%%%%%%%%%%%%%%%%%%%%%

\recipe{Spaghetti with Nuts and Dried Fruit}
% \italianrecipe{'O Sicchio d'a Munnezza}
\italianrecipealtsort{'O Sicchio d'a Munnezza}{O Sicchio d'a Munnezza}
\vegan
\region{Campania}
\ingredient{Raisins}
\ingredient{Nuts}
\ingredient{Olives, Black}
\ingredient{Tomatoes}

\newpage	%105 %%%%%%%%%%%%%%%%%%%%%%%%%%%%%%%%%%%%%%%%%%%%%%%%%%%%%%%%%%%%%%%%%%

\photo{Spaghetti with Nuts and Dried Fruit}

\newpage	%106 %%%%%%%%%%%%%%%%%%%%%%%%%%%%%%%%%%%%%%%%%%%%%%%%%%%%%%%%%%%%%%%%%%

\recipe{Pasta with Thick Tomato Sauce}
\italianrecipe{Pasta con Salsa Calabra}
\vegan
\region{Calabria}
\ingredient{Tomatoes}

\newpage	%107 %%%%%%%%%%%%%%%%%%%%%%%%%%%%%%%%%%%%%%%%%%%%%%%%%%%%%%%%%%%%%%%%%%

\recipe{Pasta with Eggplant and Ricotta Salata}
\italianrecipe{Norma}
\region{Sicilia}
\ingredient{Aubergines}
\ingredient{Tomatoes, Plum}
\ingredient{Cheese, Ricotta}	% Is Ricotta Salata different from Ricotta?

\newpage	%108 %%%%%%%%%%%%%%%%%%%%%%%%%%%%%%%%%%%%%%%%%%%%%%%%%%%%%%%%%%%%%%%%%%

\recipe{Egg Noodles with Olive Oil and Breadcrumbs}
\italianrecipe{Curzul Olio e Pangrattato}
\region{Emilia-Romagna}
% We have a hidden egg noodle recipe here
\refcurrentrecipe{Curzul}
\refcurrentrecipe{Egg Noodles}
% and then no real ingredients - oil and breadcrumbs...
\ingredient{Breadcrumbs}

\newpage	%109 %%%%%%%%%%%%%%%%%%%%%%%%%%%%%%%%%%%%%%%%%%%%%%%%%%%%%%%%%%%%%%%%%%

\recipestart{Thin Egg Noodles with Saffron Sauce}
\italianrecipe{Chitarrina allo Zafferano}
\region{Abruzzo}
% Another hidden pasta recipe
\refcurrentrecipe{Chitarrina}	% thin square pasta noodles
% There are many ingredients in the broth - hard to decide which are critical
\ingredient{Potatoes}
\ingredient{Saffron}
\ingredient{Cream}

\newpage	%110 %%%%%%%%%%%%%%%%%%%%%%%%%%%%%%%%%%%%%%%%%%%%%%%%%%%%%%%%%%%%%%%%%%

% The above recipe spans 2 pages!
\recipestop{Thin Egg Noodles with Saffron Sauce}

\recipe{Noodles with Oxtongue Greens and Prugnolo Mushrooms}
\italianrecipe{Tagliatelle con Aspraggine e Prugnoli}
\vegan
\region{Molise}
\ingredient{Bristly Oxtongue}	% A weed. Just 'oxtongue' in the book. Like Dandelion
\ingredient{Mushrooms, Prugnolo}

\newpage	%111 %%%%%%%%%%%%%%%%%%%%%%%%%%%%%%%%%%%%%%%%%%%%%%%%%%%%%%%%%%%%%%%%%%

\recipe{Garlicky Egg Pasta}
\italianrecipe{Pici all'aglione}
\region{Toscana}
\ingredient{Eggs}
\refcurrentrecipe{Pici}	% Hand rolled round pasta

\newpage	%112 %%%%%%%%%%%%%%%%%%%%%%%%%%%%%%%%%%%%%%%%%%%%%%%%%%%%%%%%%%%%%%%%%%

\recipe{Orecchiette with Broccoli Rabe}
\italianrecipe{Orecchiette con Cime di Rapa}
\region{Puglia}
\ingredient{Broccoli, Rabe}

\newpage	%113 %%%%%%%%%%%%%%%%%%%%%%%%%%%%%%%%%%%%%%%%%%%%%%%%%%%%%%%%%%%%%%%%%%

\photo{Orecchiette with Broccoli Rabe}

\newpage	%114 %%%%%%%%%%%%%%%%%%%%%%%%%%%%%%%%%%%%%%%%%%%%%%%%%%%%%%%%%%%%%%%%%%

\recipe{Green and Yellow Noodles with Romagna Shallots}
\italianrecipe{Paglia e Fieno allo Scalogno di Romagna}
\region{Emilia-Romagna}
\ingredient{Spinach}	% actually go in the pasta
\refcurrentrecipe{Paglia e Fieno}	% 'Straw and Hay' pasta
\ingredient{Tomatoes}
\ingredient{Shallots}

\newpage	%115 %%%%%%%%%%%%%%%%%%%%%%%%%%%%%%%%%%%%%%%%%%%%%%%%%%%%%%%%%%%%%%%%%%

\recipe{Thick Handmade Spaghetti with Black Truffle}
\italianrecipe{Umbricelli al Tartufo Nero}
\region{Umbria}
\refcurrentrecipe{Umbricelli}	% thick spaghetti
\ingredient{Mushrooms, Porcini}
\ingredient{Truffles, Black}

\newpage	%116 %%%%%%%%%%%%%%%%%%%%%%%%%%%%%%%%%%%%%%%%%%%%%%%%%%%%%%%%%%%%%%%%%%

\recipe{Handmade Cocoa Powder Pasta}
\italianrecipe{Maccheroni Dolci}
\region{Umbria}
\refcurrentrecipe{Maccheroni}	% Maccaroni!
\ingredient{Cocoa Powder}

\newpage	%117 %%%%%%%%%%%%%%%%%%%%%%%%%%%%%%%%%%%%%%%%%%%%%%%%%%%%%%%%%%%%%%%%%%

\recipe{Yeasted Noodles in Tomato Sauce}
\italianrecipe{Pencianelle al Pomdoro}
\region{Marche}
\refcurrentrecipe{Pencianelle}
% We use Tinned rather than Canned
\ingredient{Tomatoes, Tinned}

\newpage	%118 %%%%%%%%%%%%%%%%%%%%%%%%%%%%%%%%%%%%%%%%%%%%%%%%%%%%%%%%%%%%%%%%%%

\recipe{Handmade Semolina Pasta with Fava Beans}
\italianrecipe{Lolli con le Fave}
\vegan
\region{Sicilia}
\refcurrentrecipe{Lolli}
\ingredient{Beans, Fava}

\newpage	%119 %%%%%%%%%%%%%%%%%%%%%%%%%%%%%%%%%%%%%%%%%%%%%%%%%%%%%%%%%%%%%%%%%%

\photo{Handmade Semolina Pasta with Fava Beans}

\newpage	%120 %%%%%%%%%%%%%%%%%%%%%%%%%%%%%%%%%%%%%%%%%%%%%%%%%%%%%%%%%%%%%%%%%%

\recipe{Egg Semolina Pasta with Zucchini and Truffle}
\italianrecipe{Strangozzi con Zucchine, Ricotta e Tartufo}
\region{Umbria}
\refcurrentrecipe{Strangozzi}
\ingredient{Zucchini}
\ingredient{Cheese, Ricotta}
\ingredient{Truffles}

\newpage	%121 %%%%%%%%%%%%%%%%%%%%%%%%%%%%%%%%%%%%%%%%%%%%%%%%%%%%%%%%%%%%%%%%%%

\recipe{Egg Semolina Pasta with Ground Pepper Powder}
\italianrecipe{Cannerozzi allo Zafarano}
\region{Basilicata}
\refcurrentrecipe{Cannerozzi}
\ingredient{Paprika}	% Because I doubt we can get Senise Pepper
\ingredient{Cheese, Ricotta}

\newpage	%122 %%%%%%%%%%%%%%%%%%%%%%%%%%%%%%%%%%%%%%%%%%%%%%%%%%%%%%%%%%%%%%%%%%

\recipe{Egg Semolina Pasta with Wild Cardoon Buds}
\italianrecipe{Maccaruni con i Cacocciuli}
\region{Calabria}
\refcurrentrecipe{Maccaruni}
\ingredient{Cardoons}

\newpage	%123 %%%%%%%%%%%%%%%%%%%%%%%%%%%%%%%%%%%%%%%%%%%%%%%%%%%%%%%%%%%%%%%%%%

\recipe{Egg Yolk Semolina Pasta with Bryony Shoots}
\italianrecipe{Tonnacchioli con le Sparacogne}
\region{Sicilia}
\refcurrentrecipe{Tonnacchioli}
\ingredient{Bryony, Shoots}

\newpage	%124 %%%%%%%%%%%%%%%%%%%%%%%%%%%%%%%%%%%%%%%%%%%%%%%%%%%%%%%%%%%%%%%%%%

% Oh yum. Not long before getting the book, I had hunted down this recipe!
\recipe{Buckwheat Noodles with Cheese}
\italianrecipe{Pizzoccheri}
\region{Lombardia}
% These are the key - buckwheat, potato, cabbage and cheese!
\ingredient{Buckwheat Flour}	% It is a novel ingredient, so index it
\ingredient{Potatoes}
\ingredient{Cabbage, Savoy}
\ingredient{Cheese, Casera}

\newpage	%125 %%%%%%%%%%%%%%%%%%%%%%%%%%%%%%%%%%%%%%%%%%%%%%%%%%%%%%%%%%%%%%%%%%

\photo{Buckwheat Noodles with Cheese}

\newpage	%126 %%%%%%%%%%%%%%%%%%%%%%%%%%%%%%%%%%%%%%%%%%%%%%%%%%%%%%%%%%%%%%%%%%

\recipe{Orecchiette with Breadcrumbs and Dried Peppers}
\italianrecipe{Orecchiette Mollicate}
\region{Basilicata}
\refcurrentrecipe{Orecchiette}
\ingredient{Peppers, Senise}
% Not bothering to add the Tomatoes or Cheese (yet?)

\newpage	%127 %%%%%%%%%%%%%%%%%%%%%%%%%%%%%%%%%%%%%%%%%%%%%%%%%%%%%%%%%%%%%%%%%%

\recipe{Fregola Pasta with Artichokes and Wild Cardoons}
\italianrecipe{Fregola con Carciofi e Cardi Selvatici}
\vegan
\region{Sardegna}
\refcurrentrecipe{Fregola}
\ingredient{Artichokes, Globe}
\ingredient{Cardoons}

\newpage	%128 %%%%%%%%%%%%%%%%%%%%%%%%%%%%%%%%%%%%%%%%%%%%%%%%%%%%%%%%%%%%%%%%%%

\recipe{Handmade Semolina Pasta with Potatoes and Tomatoes}
\italianrecipe{Macarrones de Ortu}
\region{Sardegna}
\refcurrentrecipe{Macarrones}
\ingredient{Tomatoes}
\ingredient{Potatoes}

\newpage	%129 %%%%%%%%%%%%%%%%%%%%%%%%%%%%%%%%%%%%%%%%%%%%%%%%%%%%%%%%%%%%%%%%%%

\recipe{Twisted Handmade Semolina Pasta with Tomatoes}
\italianrecipe{Sagne 'ncannulate con Pomodoro}
\region{Puglia}
\refcurrentrecipe{Sagne 'ncannulate}
\ingredient{Tomatoes, Cherry}

\newpage	%130 %%%%%%%%%%%%%%%%%%%%%%%%%%%%%%%%%%%%%%%%%%%%%%%%%%%%%%%%%%%%%%%%%%

\recipe{Bigoli with Wild Thistle and Oyster Mushrooms}
\italianrecipe{Bigoli con Cardoncino e Cardoncelli}
\region{Puglia}
\ingredient{Wild Thistles}
\ingredient{Mushrooms, Oyster}

\newpage	%131 %%%%%%%%%%%%%%%%%%%%%%%%%%%%%%%%%%%%%%%%%%%%%%%%%%%%%%%%%%%%%%%%%%

\recipe{Buckwheat Pasta with Aromatic Herbs}
\italianrecipe{Fregnacce con Verdure}
\region{Lazio}
\refcurrentrecipe{Fregnacce}
\ingredient{Buckwheat Flour}

\newpage	%132 %%%%%%%%%%%%%%%%%%%%%%%%%%%%%%%%%%%%%%%%%%%%%%%%%%%%%%%%%%%%%%%%%%

\recipe{Green Agnolotti}
\italianrecipe{Agnolotti Verde}
\region{Piemonte}
\ingredient{Chard}
\ingredient{Rice}
\ingredient{Cabbage, Savoy}
\ingredient{Endive}

\newpage	%133 %%%%%%%%%%%%%%%%%%%%%%%%%%%%%%%%%%%%%%%%%%%%%%%%%%%%%%%%%%%%%%%%%%

\photo{Green Agnolotti}

\newpage	%134 %%%%%%%%%%%%%%%%%%%%%%%%%%%%%%%%%%%%%%%%%%%%%%%%%%%%%%%%%%%%%%%%%%

\recipe{Cornmeal Pasta with Spicy Artichokes}
\italianrecipe{Cresc' Tajat ai Carciofi}
\region{March}
\ingredient{Polenta}
\ingredient{Artichokes, Globe}
\ingredient{Tomatoes}

\newpage	%135 %%%%%%%%%%%%%%%%%%%%%%%%%%%%%%%%%%%%%%%%%%%%%%%%%%%%%%%%%%%%%%%%%%

\recipe{Chestnut Noodles with Porcini Mushrooms}
\italianrecipe{Tagliatelle Piccanti di Castagne}
\vegan
\region{Campania}
\ingredient{Chestnut Flour}
\ingredient{Mushrooms, Porcini}

\newpage	%136 %%%%%%%%%%%%%%%%%%%%%%%%%%%%%%%%%%%%%%%%%%%%%%%%%%%%%%%%%%%%%%%%%%

\recipe{Chestnut Noodles with Walnuts and Cocoa}
\italianrecipe{Tagliatelle di Castagne con Noci e Cacao}
\region{Campania}
\ingredient{Nuts, Walnuts}
\ingredient{Chestnut Flour}
\ingredient{Cocoa Powder}

\newpage	%137 %%%%%%%%%%%%%%%%%%%%%%%%%%%%%%%%%%%%%%%%%%%%%%%%%%%%%%%%%%%%%%%%%%

\recipe{Carob Pasta with Pistachios}
\italianrecipe{Pasta di Carrube al Sugo di Pistacchi}
\region{Sicilia}
\ingredientredundant{Carob Powder}
\ingredient{Nuts, Pistachios}

\newpage	%138 %%%%%%%%%%%%%%%%%%%%%%%%%%%%%%%%%%%%%%%%%%%%%%%%%%%%%%%%%%%%%%%%%%

\recipe{Cavatelli with Semolina, Barley, Chickpea and Fava Flours}
\italianrecipe{Cavatelli a Mischiglio}
\vegan
\region{Basilicata}
\refcurrentrecipe{Cavatelli}
\ingredient{Semolina, Flour}
\ingredient{Barley, Flour}
\ingredient{Chickpea Flour}
\ingredient{Semolina, Flour}

\newpage	%139 %%%%%%%%%%%%%%%%%%%%%%%%%%%%%%%%%%%%%%%%%%%%%%%%%%%%%%%%%%%%%%%%%%

\recipe{Ravioli with Ramsons, Nettles and Wild Spinach in Butter Sauce}
\italianrecipe{Ravioli di Erbe Selvatiche al Burro}
\region{Alto Adige}
\ingredient{Ramsons}
\ingredient{Nettles}
\ingredient{Lambsquaters}	% also known as fat hen, wild spinach etc.

\newpage	%140 %%%%%%%%%%%%%%%%%%%%%%%%%%%%%%%%%%%%%%%%%%%%%%%%%%%%%%%%%%%%%%%%%%

\recipe{Potato Dough Ravioli with Wild Greens and Herbs}
\italianrecipe{Ravioli di Patate con Erbette}
\region{Alto Adige}
% Recipe is non-specific
\ingredient{Wild Greens}
\ingredient{Potatoes}

\newpage	%141 %%%%%%%%%%%%%%%%%%%%%%%%%%%%%%%%%%%%%%%%%%%%%%%%%%%%%%%%%%%%%%%%%%

\photo{Potato Dough Ravioli with Wild Greens and Herbs}

\newpage	%142 %%%%%%%%%%%%%%%%%%%%%%%%%%%%%%%%%%%%%%%%%%%%%%%%%%%%%%%%%%%%%%%%%%

\recipe{Ravioli with Dried Pears and Ricotta}
\italianrecipe{Klotznnudin}	% how non-Italian :-)
\region{Friuli Venezia Giulia}
\ingredient{Pears, Dried}
\ingredient{Cheese, Ricotta}

\newpage	%143 %%%%%%%%%%%%%%%%%%%%%%%%%%%%%%%%%%%%%%%%%%%%%%%%%%%%%%%%%%%%%%%%%%

\recipe{Potato and Mint Stuffed Pasta}
\italianrecipe{Culurzones di Patate e Menta}
\region{Sardegna}
\refcurrentrecipe{Culurzones}	% Look them up - they are so pretty!
\ingredient{Potatoes}
\ingredient{Mint}	% apparently, can also use Sage
\ingredient{Cheese, Sheep's}

\newpage	%144 %%%%%%%%%%%%%%%%%%%%%%%%%%%%%%%%%%%%%%%%%%%%%%%%%%%%%%%%%%%%%%%%%%

\recipe{Cheese Stuffed Pasta in Walnut Sauce}
\italianrecipe{Pansotti con Salsa di Noci}
\region{Liguria}
\refcurrentrecipe{Pansotti}
\ingredient{Cheese}	% Many sorts...
\ingredient{Nuts, Walnuts}

\newpage	%145 %%%%%%%%%%%%%%%%%%%%%%%%%%%%%%%%%%%%%%%%%%%%%%%%%%%%%%%%%%%%%%%%%%

\recipe{Tortelli with Spinach and Cheese}
\italianrecipe{Canc\'{i}}
\region{Alto Adige}
\ingredient{Spinach}
\ingredient{Cheese, Ricotta}

\newpage	%146 %%%%%%%%%%%%%%%%%%%%%%%%%%%%%%%%%%%%%%%%%%%%%%%%%%%%%%%%%%%%%%%%%%

\recipe{Tail Tortelli}	"Tail" in the book - don't need the quotes for the index
\italianrecipe{Tortelli con la Coda}
\region{Emilia-Romagna}
\ingredient{Spinach}
\ingredient{Cheese, Ricotta}

\newpage	%147 %%%%%%%%%%%%%%%%%%%%%%%%%%%%%%%%%%%%%%%%%%%%%%%%%%%%%%%%%%%%%%%%%%

\recipe{Squash Tortelli}
\italianrecipe{Tortelli di Zucca}
\region{Lombardi}
\ingredient{Squash, Mantova}	% Or Kobucha - probably easier to get

\newpage	%148 %%%%%%%%%%%%%%%%%%%%%%%%%%%%%%%%%%%%%%%%%%%%%%%%%%%%%%%%%%%%%%%%%%

\recipe{Ricotta Ravioli with Sugar and Cinnamon}
\italianrecipe{Ravioli di Ricotta con Zucchero e Cannella}
\region{Abruzzo}
\ingredient{Cheese, Ricotta}
\ingredient{Cinnamon}

\newpage	%149 %%%%%%%%%%%%%%%%%%%%%%%%%%%%%%%%%%%%%%%%%%%%%%%%%%%%%%%%%%%%%%%%%%

\photo{Ricotta Ravioli with Sugar and Cinnamon}

\newpage	%150 %%%%%%%%%%%%%%%%%%%%%%%%%%%%%%%%%%%%%%%%%%%%%%%%%%%%%%%%%%%%%%%%%%

\recipe{Rye Ravioli with Wild Spinach}
\italianrecipe{Cajoncie di Moena}
\region{Trentino}
\ingredient{Spinach, Wild}

\newpage	%151 %%%%%%%%%%%%%%%%%%%%%%%%%%%%%%%%%%%%%%%%%%%%%%%%%%%%%%%%%%%%%%%%%%

\recipe{Rye Spinach Tortelli}
\italianrecipe{Schlutzkrapfen}
\region{Alto Adige}
\ingredient{Spinach}
% As previous recipe, also has cheese (Ricotta, Mascapone etc.), but do we need to list
% that as a key ingredient - probably not.

\newpage	%152 %%%%%%%%%%%%%%%%%%%%%%%%%%%%%%%%%%%%%%%%%%%%%%%%%%%%%%%%%%%%%%%%%%

\recipe{Leek Ravioli Cooked on a Griddle}
\italianrecipe{Tultei}
\region{Piemonte}
\ingredient{Leeks}
\ingredient{Potatoes}
\ingredient{Wild Greens}	% Any greens - Spinach, Chard etc.
% 'Bruzzo' - sour ricotta, for serving. Don't really need to list.
% This recipe reads a bit odd - simmer the leeks in milk for an hour? Boil the
% peeled potatoes for **an hour** ?? That just sounds like waaaay too long...

\newpage	%153 %%%%%%%%%%%%%%%%%%%%%%%%%%%%%%%%%%%%%%%%%%%%%%%%%%%%%%%%%%%%%%%%%%

\recipe{Vegetable Lasagna}
\italianrecipe{Lasagna Verde}
\region{Lombardia}
% Hmm, nothing standing out really as a key ingredient! Some interesting ingredients
% though - raisins, pine nuts, fontina, curry powder!
% There is a hidden recipe here though...
\refcurrentrecipealtsort{B\'{e}chamel}{Bechamel}

\newpage	%154 %%%%%%%%%%%%%%%%%%%%%%%%%%%%%%%%%%%%%%%%%%%%%%%%%%%%%%%%%%%%%%%%%%

\recipe{Cardoon Cannelloni}
\italianrecipe{Cannelloni di Cardi}
\region{Piemonte}
\ingredient{Cardoons}
\ingredient{Cheese, Ricotta}	% Got to have Ricotta in a Cannelloni...

\newpage	%155 %%%%%%%%%%%%%%%%%%%%%%%%%%%%%%%%%%%%%%%%%%%%%%%%%%%%%%%%%%%%%%%%%%

\recipe{Potato Gnocchi with Cheese and Butter}
\italianrecipe{Gnocchi di Patate alla Bava}
\region{Valle d'Aosta}
\ingredient{Potatoes}
\ingredient{Cheese, Fontina}
\refcurrentrecipe{Gnocchi}

\newpage	%156 %%%%%%%%%%%%%%%%%%%%%%%%%%%%%%%%%%%%%%%%%%%%%%%%%%%%%%%%%%%%%%%%%%

% Oh god, this recipe spreads 3 pages, with the photo page in the middle!
\recipestart{Cheese and Walnut Stuffed Agnolotti}
\italianrecipe{Agnolotti con Formadi Frant}
\region{Friuli Venezia Giulia}
\ingredient{Potatoes}
\ingredient{Cheese, Formadi Frant}	% Good luck finding that! ;-)

\newpage	%157 %%%%%%%%%%%%%%%%%%%%%%%%%%%%%%%%%%%%%%%%%%%%%%%%%%%%%%%%%%%%%%%%%%

\photo{Cheese and Walnut Stuffed Agnolotti}

\newpage	%158 %%%%%%%%%%%%%%%%%%%%%%%%%%%%%%%%%%%%%%%%%%%%%%%%%%%%%%%%%%%%%%%%%%

\recipestop{Cheese and Walnut Stuffed Agnolotti}

\recipe{Potato Dumplings in Garlic Butter}
\italianrecipe{Coj\"{e}tte al Seirass}
\region{Piemonte}
\ingredient{Potatoes}
\ingredient{Cheese, Ricotta}	% really Seirass, but probably cannot get that

\newpage	%159 %%%%%%%%%%%%%%%%%%%%%%%%%%%%%%%%%%%%%%%%%%%%%%%%%%%%%%%%%%%%%%%%%%

\recipe{Rich Potato Dumplings}
\italianrecipe{Ravioles}
\region{Piemonte}
\ingredient{Potatoes}
\ingredient{Cheese, Fontina}	% Substitue for the specified Toma di Montagna

\newpage	%160 %%%%%%%%%%%%%%%%%%%%%%%%%%%%%%%%%%%%%%%%%%%%%%%%%%%%%%%%%%%%%%%%%%

\recipe{Stuffed Potato Dumplings}
\italianrecipe{Struki Lessi}
\region{Friuli Venezia Giulia}
\ingredient{Potatoes}

\newpage	%161 %%%%%%%%%%%%%%%%%%%%%%%%%%%%%%%%%%%%%%%%%%%%%%%%%%%%%%%%%%%%%%%%%%

\recipe{Potato Plum Dumplings}
\italianrecipe{Gnocchi di Susine}
\region{Friuli Venezia Giulia}
\ingredient{Potatoes}
\ingredient{Plums}

\newpage	%162 %%%%%%%%%%%%%%%%%%%%%%%%%%%%%%%%%%%%%%%%%%%%%%%%%%%%%%%%%%%%%%%%%%

\recipe{Spinach Dumplings}
\italianrecipe{Strangolapreti}
\region{Trentino}
\ingredient{Bread}
\ingredient{Spinach}

\newpage	%163 %%%%%%%%%%%%%%%%%%%%%%%%%%%%%%%%%%%%%%%%%%%%%%%%%%%%%%%%%%%%%%%%%%

\recipe{Spinach and Potato Dumplings}
\italianrecipe{Chicchere di Spinaci e Patate}
\region{Toscana}
\ingredient{Spinach}
\ingredient{Potatoes}

\newpage	%164 %%%%%%%%%%%%%%%%%%%%%%%%%%%%%%%%%%%%%%%%%%%%%%%%%%%%%%%%%%%%%%%%%%

\recipe{Cheese Cappellacci}
\italianrecipe{Cappellacci di Magro}
\region{Emilia-Romagna}
% No leap out key ingredients

\newpage	%165 %%%%%%%%%%%%%%%%%%%%%%%%%%%%%%%%%%%%%%%%%%%%%%%%%%%%%%%%%%%%%%%%%%

\photo{Cheese Cappellacci}

\newpage	%166 %%%%%%%%%%%%%%%%%%%%%%%%%%%%%%%%%%%%%%%%%%%%%%%%%%%%%%%%%%%%%%%%%%

\recipe{Bread and Spinach Dumplings}
\italianrecipe{Strangulaprievete Verdi}
\region{Basilicata}
\ingredient{Spinach}
\ingredient{Bread}

\newpage	%167 %%%%%%%%%%%%%%%%%%%%%%%%%%%%%%%%%%%%%%%%%%%%%%%%%%%%%%%%%%%%%%%%%%

\recipe{Spinach and Cheese Dumplings}
\italianrecipe{Malfatti di Bag\'{o}ss di Alpeggio}
\region{Lombardia}
\ingredient{Spinach}
\ingredient{Cheese, Cows}	% Unlikely to find Bagoss

\newpage	%168 %%%%%%%%%%%%%%%%%%%%%%%%%%%%%%%%%%%%%%%%%%%%%%%%%%%%%%%%%%%%%%%%%%

\recipe{Green Gnocchi with Aged Cheese}
\italianrecipe{Gnocc de la Cua}
\region{Lombardia}
\ingredient{Wild Greens}
\ingredient{Cheese, Parmesan}	% Recipe suggests this as alternative
\ingredient{Potatoes}

\newpage	%169 %%%%%%%%%%%%%%%%%%%%%%%%%%%%%%%%%%%%%%%%%%%%%%%%%%%%%%%%%%%%%%%%%%

\recipe{Green Flour Dumplings}
\italianrecipe{Fregoloz alle Erbe}
\region{Friuli Venezia Giulia}
\ingredient{Wild Greens}

\newpage	%170 %%%%%%%%%%%%%%%%%%%%%%%%%%%%%%%%%%%%%%%%%%%%%%%%%%%%%%%%%%%%%%%%%%

\recipe{Squash Gnocchi with Smoked Ricotta}
\italianrecipe{Gnocchi di Zucca con Ricotta Affumicata}
\region{Veneto}
\ingredient{Squash, Butternut}
\ingredient{Cheese, Ricotta, Smoked}

\newpage	%171 %%%%%%%%%%%%%%%%%%%%%%%%%%%%%%%%%%%%%%%%%%%%%%%%%%%%%%%%%%%%%%%%%%

\recipe{Vegetable Dumplings}
\italianrecipe{Gnudi dell'orto}
\region{Toscana}
\ingredient{Zucchini}
\ingredient{Carrots}

\newpage	%172 %%%%%%%%%%%%%%%%%%%%%%%%%%%%%%%%%%%%%%%%%%%%%%%%%%%%%%%%%%%%%%%%%%

\recipe{Nettle Dumplings with Mushroom Sauce}
\italianrecipe{Gnochetti di Ortiche con Salsa di Funghi}
\region{Valle d'Aosta}
\ingredient{Potatoes}
\ingredient{Nettles}
\ingredient{Mushrooms, Porcini}

\newpage	%173 %%%%%%%%%%%%%%%%%%%%%%%%%%%%%%%%%%%%%%%%%%%%%%%%%%%%%%%%%%%%%%%%%%

\photo{Nettle Dumplings with Mushroom Sauce}

\newpage	%174 %%%%%%%%%%%%%%%%%%%%%%%%%%%%%%%%%%%%%%%%%%%%%%%%%%%%%%%%%%%%%%%%%%

\recipe{Chestnut Gnocchi in Cheese Cups}
\italianrecipe{Gnocchi di Castagne in Cialda}
\region{Valle d'Aosta}
\refcurrentrecipe{Chestnuts, Glazed}
\ingredient{Chestnuts}
\ingredient{Potatoes}
\ingredient{Fonduta}
\ingredient{Truffles, White}	% hmmmmm

\newpage	%175 %%%%%%%%%%%%%%%%%%%%%%%%%%%%%%%%%%%%%%%%%%%%%%%%%%%%%%%%%%%%%%%%%%

\recipe{Polenta Gnocchi with Porcini Mushrooms}
\italianrecipe{Gnocchetti di Polenta con Porcini}
\region{Valle d'Aosta}
\ingredient{Mushrooms, Porcini}
\ingredient{Polenta, Cooked}

\newpage	%176 %%%%%%%%%%%%%%%%%%%%%%%%%%%%%%%%%%%%%%%%%%%%%%%%%%%%%%%%%%%%%%%%%%

\recipe{Buckwheat Dumplings with Puzzone di Moena Cheese}
\italianrecipe{Canederli di Puzzone di Moena}
\region{Alto Adige}
\ingredient{Leeks}
\ingredient{Bread, Buckwheat}	% Never seen this?
\ingredient{Cheese, Moena}	% I reckon mozzarella would work... but not as stinky
\ingredient{Wine, Red}
\ingredient{Potatoes}

\newpage	%177 %%%%%%%%%%%%%%%%%%%%%%%%%%%%%%%%%%%%%%%%%%%%%%%%%%%%%%%%%%%%%%%%%%

\recipe{Baked Semolina Dumplings}
\italianrecipe{Gnocchi alla Romana}
\region{Lazio}
\ingredient{Semolina, Flour}
\ingredient{Polenta, Cooked}

\newpage	%178 %%%%%%%%%%%%%%%%%%%%%%%%%%%%%%%%%%%%%%%%%%%%%%%%%%%%%%%%%%%%%%%%%%

\recipe{Bread Dumplings with Potato}
\italianrecipe{Gnocchi di Colla}
\region{Lombardia}
\ingredient{Bread, Stale}
\ingredient{Potatoes}

\newpage	%179 %%%%%%%%%%%%%%%%%%%%%%%%%%%%%%%%%%%%%%%%%%%%%%%%%%%%%%%%%%%%%%%%%%

\recipe{Whole Wheat Bread Dumplings with Cheese}
\italianrecipe{Canederli ai Formaggi}
\region{Alto Adige}
\ingredient{Cheese, Malga}
\ingredient{Bread, Whole Wheat}

\newpage	%180 %%%%%%%%%%%%%%%%%%%%%%%%%%%%%%%%%%%%%%%%%%%%%%%%%%%%%%%%%%%%%%%%%%

\recipe{Green Bread Dumplings}
\italianrecipe{Gnocchi Verdi}
\region{Trentino}
\ingredient{Bread, Stale Sourdough}
\ingredient{Chard}	% or Spinach

\newpage	%181 %%%%%%%%%%%%%%%%%%%%%%%%%%%%%%%%%%%%%%%%%%%%%%%%%%%%%%%%%%%%%%%%%%

\recipe{Grated Bread Dumplings}
\italianrecipe{Gratini}
\region{Trentino}
\ingredient{Breadcrumbs}

%%%%%%%%%%%%%%%%%%%%%%%%%%%%% END OF PASTA %%%%%%%%%%%%%%%%%%%%%%%%%%%%%%%%%%%%%

%%%%%%%%%%%%%%%%%%%%%%%%%%%%% START OF RICE %%%%%%%%%%%%%%%%%%%%%%%%%%%%%%%%%%%%%

% Technically 'Rice and other Grains'

% First recipe is on page 184 - set to 183, and then bump to new page

\setcounter{page}{183} %183 %%%%%%%%%%%%%%%%%%%%%%%%%%%%%%%%%%%%%%%%%%%%%%%%%%%%%%%%%%%%%%%%%%

\newpage	%184 %%%%%%%%%%%%%%%%%%%%%%%%%%%%%%%%%%%%%%%%%%%%%%%%%%%%%%%%%%%%%%%%%%
\category{Rice and Grains}

\recipe{Barolo Risotto}
\italianrecipe{Risotto al Barolo}
\region{Piemonte}
\ingredient{Wine, Barolo}	% Heh, if I have Barolo, it's not going in the food!
\ingredient{Tomatoes}

\newpage	%185 %%%%%%%%%%%%%%%%%%%%%%%%%%%%%%%%%%%%%%%%%%%%%%%%%%%%%%%%%%%%%%%%%%

\recipe{Fennel Risotto}
\italianrecipe{Risotto coi Fenoci}
\region{Veneto}
\ingredient{Fennel, Bulb}

\recipe{Spicy Risotto}
\italianrecipe{Risotto Piccante}
\region{Calabria}
\ingredient{Peppers, Chilli}

\newpage	%186 %%%%%%%%%%%%%%%%%%%%%%%%%%%%%%%%%%%%%%%%%%%%%%%%%%%%%%%%%%%%%%%%%%

\recipe{Risotto with Maidenstears}
\italianrecipe{Risotto con i Carletti}
\region{Veneto}
\ingredient{Wild Greens}	% Maidens Tears in recipe, with is Bladder Campion

\newpage	%187 %%%%%%%%%%%%%%%%%%%%%%%%%%%%%%%%%%%%%%%%%%%%%%%%%%%%%%%%%%%%%%%%%%

\recipe{Risotto with Pears and Carmasiano Cheese}
\italianrecipe{Risotto Carmasciano pere e Mosto Cotto}
\region{Campania}
\ingredient{Pears}
\ingredient{Cheese, Sheep's}
\ingredient{Mosto Cotto}	% Reduced grape juice

\newpage	%188 %%%%%%%%%%%%%%%%%%%%%%%%%%%%%%%%%%%%%%%%%%%%%%%%%%%%%%%%%%%%%%%%%%

\recipe{Rice with Peas}
\italianrecipe{Risi e Bisi}
\region{Veneto}
\ingredient{Peas}

\newpage	%189 %%%%%%%%%%%%%%%%%%%%%%%%%%%%%%%%%%%%%%%%%%%%%%%%%%%%%%%%%%%%%%%%%%

\recipe{Rice with Asparagus and Eggs}
\italianrecipe{Riso con Asparagi e Uova}
\region{Veneto}
\ingredient{Asparagus}
\ingredient{Eggs}

\newpage	%190 %%%%%%%%%%%%%%%%%%%%%%%%%%%%%%%%%%%%%%%%%%%%%%%%%%%%%%%%%%%%%%%%%%

\recipe{Rice with Baby Asparagus and Tender Spinach}
\italianrecipe{Riso Asparagi e Spinacini}
\region{Lombardia}
\ingredient{Asparagus}
\ingredient{Spinach}

\newpage	%191 %%%%%%%%%%%%%%%%%%%%%%%%%%%%%%%%%%%%%%%%%%%%%%%%%%%%%%%%%%%%%%%%%%

\photo{Rice with Baby Asparagus and Tender Spinach}

\newpage	%192 %%%%%%%%%%%%%%%%%%%%%%%%%%%%%%%%%%%%%%%%%%%%%%%%%%%%%%%%%%%%%%%%%%

\recipe{Rice Souffl\'{e}s}
\italianrecipe{Souffl\'{e} di Riso}
\region{Piemonte}
\ingredient{Eggs}

\newpage	%193 %%%%%%%%%%%%%%%%%%%%%%%%%%%%%%%%%%%%%%%%%%%%%%%%%%%%%%%%%%%%%%%%%%

\photo{Rice Souffl\'{e}s}

\newpage	%194 %%%%%%%%%%%%%%%%%%%%%%%%%%%%%%%%%%%%%%%%%%%%%%%%%%%%%%%%%%%%%%%%%%

\recipe{Rice Dumplings}
\italianrecipe{Gnocchi di Riso}
\region{Lombardia}

\newpage	%195 %%%%%%%%%%%%%%%%%%%%%%%%%%%%%%%%%%%%%%%%%%%%%%%%%%%%%%%%%%%%%%%%%%

\recipe{Wild Rice with Cauliflower}
\italianrecipe{Riso Selvatico e Cavolfiore}
\region{Lazio}
\ingredient{Cauliflower}
\ingredient{Rice, Wild}

\newpage	%196 %%%%%%%%%%%%%%%%%%%%%%%%%%%%%%%%%%%%%%%%%%%%%%%%%%%%%%%%%%%%%%%%%%

\recipe{Risotto-Style Barley with Teroldego Wine}
\italianrecipe{Orzotto al Terodldego}
\region{Trentino}
\ingredient{Barley, Pearled}
\ingredient{Wine, Red, Teroldego}

\newpage	%197 %%%%%%%%%%%%%%%%%%%%%%%%%%%%%%%%%%%%%%%%%%%%%%%%%%%%%%%%%%%%%%%%%%

\recipe{Risotto-Style Barley with Vegetables}
\italianrecipe{Orzotto alle Verdure}
\region{Venito}
\ingredient{Barley, Pearled}
\ingredient{Beans, Borlotti}

\newpage	%198 %%%%%%%%%%%%%%%%%%%%%%%%%%%%%%%%%%%%%%%%%%%%%%%%%%%%%%%%%%%%%%%%%%

\recipe{Risotto-Style Barley with Pears and As\'{i}no Cheese}
\italianrecipe{Orzotto con Pere e As\'{i}no}
\region{Friuli Venezia Giulia}
\ingredient{Barley, Pearled}
\ingredient{Pears}
\ingredient{Cheese, As\'{i}no}	% Not sure how to get or what to substitute

\newpage	%199 %%%%%%%%%%%%%%%%%%%%%%%%%%%%%%%%%%%%%%%%%%%%%%%%%%%%%%%%%%%%%%%%%%

\recipe{Farro with Porcini}
\italianrecipe{Farrotto con Porcini}
\region{Abruzzo}
\ingredient{Mushrooms, Porcini}
% Although dropping this we drop the 'pearled' aspect from the index, we also stop
% having three consecutive lines just for this one recipe.
\ingredientredundant{Farro, Pearled}

\newpage	%200 %%%%%%%%%%%%%%%%%%%%%%%%%%%%%%%%%%%%%%%%%%%%%%%%%%%%%%%%%%%%%%%%%%

\recipe{Quinoa and Millet Croquettes}
\italianrecipe{Crocchette di Quinoa e Miglio}
\region{Lombardia}
\ingredientredundant{Quinoa}
\ingredient{Millet}

\newpage	%201 %%%%%%%%%%%%%%%%%%%%%%%%%%%%%%%%%%%%%%%%%%%%%%%%%%%%%%%%%%%%%%%%%%

\recipe{Quinoa-Stuffed Tomatoes}
\italianrecipe{Pomodori Ripieni di Quinoa}
\region{Lombardia}
\ingredientredundant{Quinoa}
\ingredient{Tomatoes}

%%%%%%%%%%%%%%%%%%%%%%%%%%%%% END OF RICE %%%%%%%%%%%%%%%%%%%%%%%%%%%%%%%%%%%%%

%%%%%%%%%%%%%%%%%%%%%%%%%%%%% START OF POLENTA %%%%%%%%%%%%%%%%%%%%%%%%%%%%%%%%%%%%%

% Technically 'Polenta and other Porridges'

% First recipe is on page 204 - set to 203, and then bump to new page

\setcounter{page}{203} %203 %%%%%%%%%%%%%%%%%%%%%%%%%%%%%%%%%%%%%%%%%%%%%%%%%%%%%%%%%%%%%%%%%%

\newpage	%204 %%%%%%%%%%%%%%%%%%%%%%%%%%%%%%%%%%%%%%%%%%%%%%%%%%%%%%%%%%%%%%%%%%
\category{Polenta and Porridges}

\recipe{Polenta with Cheese and Milk}
\italianrecipe{Polenta Grassa}
\region{Valle d'Aosta}
\ingredient{Polenta}
\ingredient{Cheese, Toma}

\recipe{Polenta with Fonduta Sauce}
\italianrecipe{Polenta con Fonduta}
\region{Valle d'Aosta}
\ingredient{Polenta}
\ingredient{Fonduta}	% Fontina cheese white sauce

\newpage	%205 %%%%%%%%%%%%%%%%%%%%%%%%%%%%%%%%%%%%%%%%%%%%%%%%%%%%%%%%%%%%%%%%%%

\recipe{Polenta with Stracchino Cheese and Black Truffle}
\italianrecipe{Polenta con Stracchino e Tartufo Nero}
\region{Lombardia}
\ingredient{Polenta}
\ingredient{Cheese, Stracchino}
\ingredient{Truffles, Black}

\newpage	%206 %%%%%%%%%%%%%%%%%%%%%%%%%%%%%%%%%%%%%%%%%%%%%%%%%%%%%%%%%%%%%%%%%%

\recipe{Extra-Creamy Polenta with Montasio cheese}
\italianrecipe{Toc' in Braide}
\region{Friuli Venezia Giulia}
\ingredient{Polenta}
\ingredient{Cheese, Montasio}

\newpage	%207 %%%%%%%%%%%%%%%%%%%%%%%%%%%%%%%%%%%%%%%%%%%%%%%%%%%%%%%%%%%%%%%%%%

\recipe{Polenta with Chicory and Cheese}
\italianrecipe{Polenta Macafana}
\region{Trentino}
\ingredient{Polenta}
\ingredient{Chicory}
\ingredient{Cheese, Spressa}

\newpage	%208 %%%%%%%%%%%%%%%%%%%%%%%%%%%%%%%%%%%%%%%%%%%%%%%%%%%%%%%%%%%%%%%%%%

\recipe{Polenta with Potatoes and Onions}
\italianrecipe{Roascht}
\vegan
\region{Alto Adige}
\ingredient{Polenta}
\ingredient{Potatoes}

\newpage	%209 %%%%%%%%%%%%%%%%%%%%%%%%%%%%%%%%%%%%%%%%%%%%%%%%%%%%%%%%%%%%%%%%%%

\recipe{Green Polenta}
\italianrecipe{Polenta con Caurigli}
\vegan
\region{Molise}
\ingredient{Polenta}
\ingredient{Broccoli}

\recipe{Polenta with Squash}
\italianrecipe{Zuf}
\region{Friuli Venezia Giulia}
\ingredient{Polenta}
\ingredient{Squash}

\newpage	%210 %%%%%%%%%%%%%%%%%%%%%%%%%%%%%%%%%%%%%%%%%%%%%%%%%%%%%%%%%%%%%%%%%%

\recipe{Chestnut Polenta with Taleggio Cheese and Cabbage}
\italianrecipe{Polenta di Castagne con Taleggio e Verza}
\region{Lombardia}
\ingredient{Polenta}
\ingredient{Chestnut Flour}
\ingredient{Cabbage, Savoy}
\ingredient{Cheese, Taleggio}

\newpage	%211 %%%%%%%%%%%%%%%%%%%%%%%%%%%%%%%%%%%%%%%%%%%%%%%%%%%%%%%%%%%%%%%%%%

\recipe{Stuffed Cabbage with Polenta}
\italianrecipe{Fagottini di Verza con Polenta}
\vegan
\region{Valle d'Aosta}
\ingredient{Polenta, Cooked}
\ingredient{Cabbage, Savoy}
\ingredient{Rice}

\newpage	%212 %%%%%%%%%%%%%%%%%%%%%%%%%%%%%%%%%%%%%%%%%%%%%%%%%%%%%%%%%%%%%%%%%%

\recipe{Wild Pea Polenta with Onions and Sheep's Cheese}
\italianrecipe{Polenta di Roveja con Cipolla e Pecorino}
\region{Marche}
\ingredient{Pea Flour}	% unclear if you can substitute Fava or Chickpea
\ingredient{Cheese, Pecorino}

\newpage	%213 %%%%%%%%%%%%%%%%%%%%%%%%%%%%%%%%%%%%%%%%%%%%%%%%%%%%%%%%%%%%%%%%%%

\photo{Wild Pea Polenta with Onions and Sheep's Cheese}

\newpage	%214 %%%%%%%%%%%%%%%%%%%%%%%%%%%%%%%%%%%%%%%%%%%%%%%%%%%%%%%%%%%%%%%%%%

\recipe{Rye Bread Polenta}
\italianrecipe{Viandon}
\region{Valle d'Aosta}
\ingredient{Bread, Rye}
\ingredient{Polenta}
\ingredient{Cheese, Fontina}

\recipe{Wheat Porridge with Wine}
\italianrecipe{Pappa Sciansca}
\region{Veneto}
\ingredient{Wine, White}
\ingredient{Polenta}
\ingredient{Cheese, Fontina}

\newpage	%215 %%%%%%%%%%%%%%%%%%%%%%%%%%%%%%%%%%%%%%%%%%%%%%%%%%%%%%%%%%%%%%%%%%

\recipe{Barley Porridge}
\italianrecipe{Peil\'{a} d'Orzo}
\region{Valle d'Aosta}
\ingredient{Potatoes}
\ingredient{Barley, Flour}
\ingredient{Cheese, Fontina}
\ingredient{Cheese, Toma}

\recipe{Semolina Porridge}
\italianrecipe{Paparotta}
\vegan
\region{Campania}
\ingredient{Semolina}
\ingredient{Peppers, Sun Dried}

%%%%%%%%%%%%%%%%%%%%%%%%%%%%% END OF POLENTA %%%%%%%%%%%%%%%%%%%%%%%%%%%%%%%%%%%%%

%%%%%%%%%%%%%%%%%%%%%%%%%%%%% START OF LEGUMES %%%%%%%%%%%%%%%%%%%%%%%%%%%%%%%%%%%%%

% First recipe is on page 218 - set to 217, and then bump to new page

\setcounter{page}{217} %217 %%%%%%%%%%%%%%%%%%%%%%%%%%%%%%%%%%%%%%%%%%%%%%%%%%%%%%%%%%%%%%%%%%

\newpage	%218 %%%%%%%%%%%%%%%%%%%%%%%%%%%%%%%%%%%%%%%%%%%%%%%%%%%%%%%%%%%%%%%%%%
\category{Legumes}

\recipe{Pureed Cicerchia Beans}
\italianrecipe{Pur\'{e} di Cicerchie di Castelvecchio}
\vegan
\region{Abruzzo}
\ingredient{Beans, Cicerchia}

\recipe{Browned Borlotti Beans and Sauerkraut}
\italianrecipe{Crauti e Fagioli}
\region{Alto Adige}
\ingredient{Beans, Borlotti}
\ingredient{Sauerkraut}

\newpage	%219 %%%%%%%%%%%%%%%%%%%%%%%%%%%%%%%%%%%%%%%%%%%%%%%%%%%%%%%%%%%%%%%%%%

\recipe{Spiced Borlotti Beans}
\italianrecipe{Fasoi in Salsa}
\vegan
\region{Veneto}
\ingredient{Beans, Borlotti}
\ingredient{Cinnamon}	% Critical - the 'spiced' element

\recipe{Beans with Sauerkraut}
\italianrecipe{Crauti Rostidi}
\vegan
\region{Trentino}
\ingredient{Beans, Borlotti}
\ingredient{Sauerkraut}

\newpage	%220 %%%%%%%%%%%%%%%%%%%%%%%%%%%%%%%%%%%%%%%%%%%%%%%%%%%%%%%%%%%%%%%%%%

\recipe{Fagiolane Beans in Barbera Wine}
\italianrecipe{Fagiolane alla Barbera}
\region{Piemonte}
\ingredient{Beans, Fagiolane}
\ingredient{Wine, Barbera}

\newpage	%221 %%%%%%%%%%%%%%%%%%%%%%%%%%%%%%%%%%%%%%%%%%%%%%%%%%%%%%%%%%%%%%%%%%

\recipe{Pasta and Lamon Beans}
\italianrecipe{Pasta e Fagioli}
\vegan
\region{Veneto}
\ingredient{Beans, Lamon}
\ingredient{Pasta, Tagliatelle}

\newpage	%222 %%%%%%%%%%%%%%%%%%%%%%%%%%%%%%%%%%%%%%%%%%%%%%%%%%%%%%%%%%%%%%%%%%

\recipe{Potato and Borlotti Bean Wedges}
\italianrecipe{Pendolon}
\region{Trentino}
\ingredient{Beans, Borlotti}
\ingredient{Potatoes}

\newpage	%223 %%%%%%%%%%%%%%%%%%%%%%%%%%%%%%%%%%%%%%%%%%%%%%%%%%%%%%%%%%%%%%%%%%

\recipe{Bean, Bread and Chard Cake}
\italianrecipe{Secra e Suriata}
\region{Calabria}
\ingredient{Beans}	% Unspecified type - Any!
\ingredient{Chard}

\newpage	%224 %%%%%%%%%%%%%%%%%%%%%%%%%%%%%%%%%%%%%%%%%%%%%%%%%%%%%%%%%%%%%%%%%%

\recipe{Thousand Things Bean Soup}
\italianrecipe{Millecosedde}
\region{Calabria}
\ingredient{Beans, Cannellini}
\ingredient{Beans, Chickpeas}
\ingredient{Beans, Cicerchia}
\ingredient{Beans, Fava}
\ingredient{Lentils}
\ingredient{Cabbage, Savoy}
\ingredient{Mushrooms, Porcini}
\ingredient{Pasta, Semolina}

\newpage	%225 %%%%%%%%%%%%%%%%%%%%%%%%%%%%%%%%%%%%%%%%%%%%%%%%%%%%%%%%%%%%%%%%%%

\recipe{Cannellini Beans with Sage}
\italianrecipe{Fagioli all'Uccelletto}
\vegan
\region{Toscana}
\ingredient{Beans, Cannellini}
\ingredient{Tomatoes, Plum}
\ingredient{Sage}

\newpage	%226 %%%%%%%%%%%%%%%%%%%%%%%%%%%%%%%%%%%%%%%%%%%%%%%%%%%%%%%%%%%%%%%%%%

\recipe{Lupini Bean Cutlets}
\italianrecipe{Cotolette di Lupini}
\vegan
\region{Basilicata}
\ingredient{Beans, Lupini}

\newpage	%227 %%%%%%%%%%%%%%%%%%%%%%%%%%%%%%%%%%%%%%%%%%%%%%%%%%%%%%%%%%%%%%%%%%

\recipe{Cannellini Beans and Chard}
\italianrecipe{Zimino di Fagioli}
\vegan
\region{Toscana}
\ingredient{Beans, Cannellini}
\ingredient{Tomatoes, Tinned}
\ingredient{Chard}

\recipe{Chickpea Pur\'{e}e}
\italianrecipe{Pur\'{e} di Ceci}
\vegan
\region{Toscana}
\ingredient{Beans, Chickpeas}

\newpage	%228 %%%%%%%%%%%%%%%%%%%%%%%%%%%%%%%%%%%%%%%%%%%%%%%%%%%%%%%%%%%%%%%%%%

\recipe{Chickpeas with Fried and Boiled Noodles}
\italianrecipe{Ciceri e Tria}
\vegan
\region{Puglia}
\ingredient{Beans, Chickpeas}
\refcurrentrecipe{Tria}
\ingredient{Semolina, Flour}

\newpage	%229 %%%%%%%%%%%%%%%%%%%%%%%%%%%%%%%%%%%%%%%%%%%%%%%%%%%%%%%%%%%%%%%%%%

\photo{Chickpeas with Fried and Boiled Noodles}

\newpage	%230 %%%%%%%%%%%%%%%%%%%%%%%%%%%%%%%%%%%%%%%%%%%%%%%%%%%%%%%%%%%%%%%%%%

\recipe{Chickpeas with Eggless Lagane Noodles}
\italianrecipe{Lagane e Ceci}
\vegan
\region{Campania}
\ingredient{Beans, Chickpeas}
\ingredient{Tomatoes}
\ingredient{Lagane}	% Semolina Noodles

\recipe{Spicy Fava Beans}
\italianrecipe{Fave 'ngrecce}
\vegan
\region{Marche}
\ingredient{Beans, Fava}

\newpage	%231 %%%%%%%%%%%%%%%%%%%%%%%%%%%%%%%%%%%%%%%%%%%%%%%%%%%%%%%%%%%%%%%%%%

\recipe{Stewed Fava Beans}
\italianrecipe{Bagiana}
\vegan
\region{Umbria}
\ingredient{Beans, Fava}

\recipe{Fava Beans with Chard}
\italianrecipe{Biete e Fave}
\vegan
\region{Abruzzo}
\ingredient{Beans, Fava}
\ingredient{Chard}

\newpage	%232 %%%%%%%%%%%%%%%%%%%%%%%%%%%%%%%%%%%%%%%%%%%%%%%%%%%%%%%%%%%%%%%%%%

\recipe{Fava Pure\'{e} with Wild Chicory}
\italianrecipe{Incapriata di Fave con Cicorielle Selvatiche}
\vegan
\region{Puglia}
\ingredient{Beans, Fava}
\ingredient{Potatoes}
\ingredient{Chicory, Wild}

\recipe{Stewed Lentils}
\italianrecipe{Lenticchie di Onano Sufate}
\vegan
\region{Lazio}
\ingredient{Lentils, Onano} % seem to be flat green lentils, a bit like our 'green lentils'

\newpage	%233 %%%%%%%%%%%%%%%%%%%%%%%%%%%%%%%%%%%%%%%%%%%%%%%%%%%%%%%%%%%%%%%%%%

\photo{Fava Pure\'{e} with Wild Chicory}

\newpage	%234 %%%%%%%%%%%%%%%%%%%%%%%%%%%%%%%%%%%%%%%%%%%%%%%%%%%%%%%%%%%%%%%%%%

\recipe{Fava Beans over Bread Cubes}
\italianrecipe{Pancotto Contadino}
\vegan
\region{Puglia}
\ingredient{Beans, Fava}
\ingredient{Potatoes}
\ingredient{Cabbage, Savoy}
\ingredient{Tomatoes}

\newpage	%235 %%%%%%%%%%%%%%%%%%%%%%%%%%%%%%%%%%%%%%%%%%%%%%%%%%%%%%%%%%%%%%%%%%

\recipe{Simmered Castelluccio Lentils}
\italianrecipe{Lenticchie di Castelluccio in Umido}
\vegan
\region{Umbria}
\ingredient{Lentils, Castelluccio}

\recipe{Lentil Pur\'{e}e}
\italianrecipe{Pur\'{e} di Lenticchie}
\region{Umbria}
\ingredient{Lentils}

%%%%%%%%%%%%%%%%%%%%%%%%%%%%% END OF LEGUMES %%%%%%%%%%%%%%%%%%%%%%%%%%%%%%%%%%%%%

%%%%%%%%%%%%%%%%%%%%%%%%%%%%% START OF EGGS %%%%%%%%%%%%%%%%%%%%%%%%%%%%%%%%%%%%%

% Technically 'Frittate, Custarts and other Egg Dishes'

% First recipe is on page 238 - set to 237, and then bump to new page
\setcounter{page}{237} %237 %%%%%%%%%%%%%%%%%%%%%%%%%%%%%%%%%%%%%%%%%%%%%%%%%%%%%%%%%%%%%%%%%%

\newpage	%238 %%%%%%%%%%%%%%%%%%%%%%%%%%%%%%%%%%%%%%%%%%%%%%%%%%%%%%%%%%%%%%%%%%
\category{Eggs}

\recipe{Yellow Pepper Frittata}
\italianrecipe{Frittata di Peperoni}
\region{Piemonte}

% A quandry - as the section is 'Eggs', all these egg based dishes will probably turn up
% in two consecutive listings in the index - a waste - we should decide later how to handle
% that ... aha, but, it might work out OK, as both the 'section' and 'ingredients' items of the
% same name look to get merged into one section in the index - so that's OK - the indexer
% automatically removes any duplication. Leave as is, as it makes it clearer in the actual
% recipes that they have eggs in...

\ingredient{Eggs}
\ingredient{Peppers, Yellow}

\newpage	%239 %%%%%%%%%%%%%%%%%%%%%%%%%%%%%%%%%%%%%%%%%%%%%%%%%%%%%%%%%%%%%%%%%%

\photo{Yellow Pepper Frittata}

\newpage	%240 %%%%%%%%%%%%%%%%%%%%%%%%%%%%%%%%%%%%%%%%%%%%%%%%%%%%%%%%%%%%%%%%%%

\recipe{Potato and Apple Frittata}
\italianrecipe{Frittata di Patate e Mele}
\region{Valle d'Aosta}
\ingredient{Eggs}
\ingredient{Potatoes}
\ingredient{Apples}

\newpage	%241 %%%%%%%%%%%%%%%%%%%%%%%%%%%%%%%%%%%%%%%%%%%%%%%%%%%%%%%%%%%%%%%%%%

\recipe{Hop Shoot Frittata}
\italianrecipe{Frittata con Bruscandoli}
\region{Veneto}
\ingredient{Eggs}
\ingredientredundant{Hop Shoots}

\recipe{Ricotta Frittata}
\italianrecipe{Froscia di Ricotta}
\region{Sicilia}
\ingredient{Eggs}
\ingredient{Cheese, Ricotta}

\newpage	%242 %%%%%%%%%%%%%%%%%%%%%%%%%%%%%%%%%%%%%%%%%%%%%%%%%%%%%%%%%%%%%%%%%%

\recipe{Wild Vitalba Frittata}
\italianrecipe{Frittata con le Vitabbie}
\region{Umbria}
\ingredient{Eggs}
\ingredient{Vitalba Sprouts}	% Old mans beard - it's a Clematis

\recipe{Pea Pod Custard}
\italianrecipe{Sformato di Bucce di Piselli}
\region{Emilia-Romagna}
\ingredient{Eggs}
\ingredientredundant{Pea Pods}

\newpage	%243 %%%%%%%%%%%%%%%%%%%%%%%%%%%%%%%%%%%%%%%%%%%%%%%%%%%%%%%%%%%%%%%%%%

\recipe{Frittata with Red Garlic}
\italianrecipe{Frittata con Aglio Rosso}
\region{Abruzzo}
\ingredient{Eggs}
\ingredient{Garlic Scapes}

\newpage	%244 %%%%%%%%%%%%%%%%%%%%%%%%%%%%%%%%%%%%%%%%%%%%%%%%%%%%%%%%%%%%%%%%%%

\recipe{Chard Frittata}
\italianrecipe{Erbazzone}
\region{Emilia-Romagna}
\ingredient{Eggs}
\ingredient{Chard}

\newpage	%245 %%%%%%%%%%%%%%%%%%%%%%%%%%%%%%%%%%%%%%%%%%%%%%%%%%%%%%%%%%%%%%%%%%

\recipe{Leek Flan with Fonduta Sauce}
\italianrecipe{Sfromatino di Porri con Fonduta}
\region{Valle d'Aosta}
\ingredient{Eggs}
\ingredient{Leeks}
\ingredient{Fonduta}

\newpage	%246 %%%%%%%%%%%%%%%%%%%%%%%%%%%%%%%%%%%%%%%%%%%%%%%%%%%%%%%%%%%%%%%%%%

\recipe{Individual Savory Herb Puddings with Aosta Blue Cheese Sauce}
\italianrecipe{Tart\'{a} alle Erbette con Salsa al Bleu d'Aosta}
\region{Valle d'Aosta}
\ingredient{Eggs}
\ingredient{Cheese, Aosta Blue}

\newpage	%247 %%%%%%%%%%%%%%%%%%%%%%%%%%%%%%%%%%%%%%%%%%%%%%%%%%%%%%%%%%%%%%%%%%

\photo{Individual Savory Herb Puddings with Aosta Blue Cheese Sauce}

\newpage	%248 %%%%%%%%%%%%%%%%%%%%%%%%%%%%%%%%%%%%%%%%%%%%%%%%%%%%%%%%%%%%%%%%%%

\recipe{Squash and Truffle Flan}
\italianrecipe{Sformato di Zucca e Tartufo}
\region{Emilia-Romagna}
\ingredient{Eggs}
\ingredient{Squash}
\ingredient{Truffles, Black}

\recipe{Jerusalem Artichoke Flans}
\italianrecipe{Flan di Topinambur}
\region{Piemonte}
\ingredient{Eggs}
\ingredient{Jerusalem Artichokes}

\newpage	%249 %%%%%%%%%%%%%%%%%%%%%%%%%%%%%%%%%%%%%%%%%%%%%%%%%%%%%%%%%%%%%%%%%%

\recipe{Individual Artichoke Flans}
\italianrecipe{Flan di Carciofi}
\region{Campania}
\ingredient{Eggs}
\ingredient{Artichokes, Globe}

\recipe{Chard with Beaten Eggs}
\italianrecipe{Coste al Brusch}
\region{Piemonte}
\ingredient{Eggs}
\ingredient{Chard}

\newpage	%250 %%%%%%%%%%%%%%%%%%%%%%%%%%%%%%%%%%%%%%%%%%%%%%%%%%%%%%%%%%%%%%%%%%

\recipe{Asparagus and Eggs}
\italianrecipe{Asparagi e Uova}
\region{Alto Adige}
\ingredient{Eggs}
\ingredient{Asparagus}

\recipe{Soft Boiled Eggs over Asparagus}
\italianrecipe{Asparagi e Uova Basocche}
\region{Friuli Venezia Giulia}
\ingredient{Eggs}
\ingredient{Asparagus}

\newpage	%251 %%%%%%%%%%%%%%%%%%%%%%%%%%%%%%%%%%%%%%%%%%%%%%%%%%%%%%%%%%%%%%%%%%

\recipe{Asparagus with Hard-Boiled Eggs and Watercress}
\italianrecipe{Asparagi con Salsa Bolzanina}
\region{Alto Adige}
\ingredient{Eggs}
\ingredient{Asparagus}

\newpage	%252 %%%%%%%%%%%%%%%%%%%%%%%%%%%%%%%%%%%%%%%%%%%%%%%%%%%%%%%%%%%%%%%%%%

\recipe{Eggs and Potatoes in Green Sauce}
\italianrecipe{Patae e Uova in Salsa Verde}
\region{Piemonte}
\ingredient{Eggs}
\ingredient{Potatoes}
\ingredient{Chard}

\recipe{Baked Eggs in Tomato Sauce}
\italianrecipe{Uova in Salsa}
\region{Piemonte}
\ingredient{Eggs}
\ingredient{Tomato Sauce}

\newpage	%253 %%%%%%%%%%%%%%%%%%%%%%%%%%%%%%%%%%%%%%%%%%%%%%%%%%%%%%%%%%%%%%%%%%

\recipe{Eggs and Artichokes in Caper and Herb Sauce}
\italianrecipe{Carciofi e Uova in Salsa Ravigotta}
\region{Lombardia}
\ingredient{Eggs}
\ingredient{Artichokes, Globe}
% \ingredient{Chard} % Capers are a key element of the sauce - but too fringe for the index?

\newpage	%254 %%%%%%%%%%%%%%%%%%%%%%%%%%%%%%%%%%%%%%%%%%%%%%%%%%%%%%%%%%%%%%%%%%

\recipe{Eggs with Provatura Cheese}
\italianrecipe{Uova alla Provatura}
\region{Lazio}
\ingredient{Eggs}
\ingredient{Cheese, Provatura}

\newpage	%255 %%%%%%%%%%%%%%%%%%%%%%%%%%%%%%%%%%%%%%%%%%%%%%%%%%%%%%%%%%%%%%%%%%

\recipe{Eggs Cutlets with Tapenade}
\italianrecipe{Cotolette di Uova}
\region{Lombardia}
\ingredient{Eggs}
\ingredient{Olives}

%%%%%%%%%%%%%%%%%%%%%%%%%%%%% END OF EGGS %%%%%%%%%%%%%%%%%%%%%%%%%%%%%%%%%%%%%

%%%%%%%%%%%%%%%%%%%%%%%%%%%%% START OF PANCAKES %%%%%%%%%%%%%%%%%%%%%%%%%%%%%%%%%%%%%

% Technically 'Pancakes, Fritters and Croquettes'

% First recipe is on page 258 - set to 257, and then bump to new page

\setcounter{page}{257} %257 %%%%%%%%%%%%%%%%%%%%%%%%%%%%%%%%%%%%%%%%%%%%%%%%%%%%%%%%%%%%%%%%%%

\newpage	%258 %%%%%%%%%%%%%%%%%%%%%%%%%%%%%%%%%%%%%%%%%%%%%%%%%%%%%%%%%%%%%%%%%%
\category{Pancakes, Fritters and Croquettes}

\recipe{Buckwheat Pancakes}
\italianrecipe{Fanzelti}
\vegan
\region{Trentino}
\ingredient{Buckwheat Flour}

\recipe{Potato Pancakes}
\italianrecipe{Subrich di Patate}
\region{Piemonte}
\ingredient{Potatoes}
\ingredient{Eggs}

\newpage	%259 %%%%%%%%%%%%%%%%%%%%%%%%%%%%%%%%%%%%%%%%%%%%%%%%%%%%%%%%%%%%%%%%%%

\recipe{Spinach Pancakes}
\italianrecipe{Frittatine di Spinaci}
\vegan
\region{Emilia-Romagna}
\ingredient{Spinach}
\ingredient{Eggs}

\newpage	%260 %%%%%%%%%%%%%%%%%%%%%%%%%%%%%%%%%%%%%%%%%%%%%%%%%%%%%%%%%%%%%%%%%%

\recipe{Chickpea Flour Pancake}
\italianrecipe{Cecina}
\vegan
\region{Toscana}
\ingredient{Chickpea Flour}

\recipe{Thin Chickpea Flour Pancake with Rosemary}
\italianrecipe{Farinata}
\vegan
\region{Liguria}
\ingredient{Chickpea Flour}

\newpage	%261 %%%%%%%%%%%%%%%%%%%%%%%%%%%%%%%%%%%%%%%%%%%%%%%%%%%%%%%%%%%%%%%%%%

\photo{Chickpea Flour Pancake}

\newpage	%262 %%%%%%%%%%%%%%%%%%%%%%%%%%%%%%%%%%%%%%%%%%%%%%%%%%%%%%%%%%%%%%%%%%

\recipe{Pancake Pasta}
\italianrecipe{Testaroli Pontremolesi}
\region{Toscana}
% No real stand out ingredients! But, there is a buried pasta recipe...
\refcurrentrecipe{Testaroli}

\newpage	%263 %%%%%%%%%%%%%%%%%%%%%%%%%%%%%%%%%%%%%%%%%%%%%%%%%%%%%%%%%%%%%%%%%%

\recipe{Potato Croquettes}
\italianrecipe{Kiffel}
\region{Venezia Giulia}
\ingredient{Potatoes}
\ingredient{Eggs}

\newpage	%264 %%%%%%%%%%%%%%%%%%%%%%%%%%%%%%%%%%%%%%%%%%%%%%%%%%%%%%%%%%%%%%%%%%

\recipe{Potato Croquettes with Marjoram and Pine Nuts}
\italianrecipe{Cuculli}
\region{Liguria}
\ingredient{Potatoes}
\ingredient{Eggs}
% Marjoram and pine nuts, although in the title, do not feel right for the index

\newpage	%265 %%%%%%%%%%%%%%%%%%%%%%%%%%%%%%%%%%%%%%%%%%%%%%%%%%%%%%%%%%%%%%%%%%

\photo{Potato Croquettes with Marjoram and Pine Nuts}

\newpage	%266 %%%%%%%%%%%%%%%%%%%%%%%%%%%%%%%%%%%%%%%%%%%%%%%%%%%%%%%%%%%%%%%%%%

\recipe{Potato and Cheese Fritters}
\italianrecipe{Crocche\'{e} di Patate}
\region{Campania}
\ingredient{Potatoes}
\ingredient{Eggs}
\ingredient{Cheese, Cilento}
\ingredient{Cheese, Provolone}

\recipe{Pan-Fried Cheese Croquettes}
\italianrecipe{Pallotte cac' e Ove}
\region{Abruzzo}
\ingredient{Bread}
\ingredient{Eggs}
\ingredient{Cheese, Pecorino}	% aged sheep, so also Manchego
% Recipe also has Parmigiano cheese - but, don't they all? :-)

\newpage	%267 %%%%%%%%%%%%%%%%%%%%%%%%%%%%%%%%%%%%%%%%%%%%%%%%%%%%%%%%%%%%%%%%%%

\photo{Potato and Cheese Fritters}

\newpage	%268 %%%%%%%%%%%%%%%%%%%%%%%%%%%%%%%%%%%%%%%%%%%%%%%%%%%%%%%%%%%%%%%%%%

\recipe{Cheese Crisp with Potatoes}
\italianrecipe{Frico con Patate}
\region{Friuli Venezia Giulia}
\ingredient{Potatoes}
\ingredient{Cheese, Montasio}

\recipe{Spinach Fritters}
\italianrecipe{Fritelle di Spinaci}
\region{Veneto}
\ingredient{Spinach}
\ingredient{Eggs}

\newpage	%269 %%%%%%%%%%%%%%%%%%%%%%%%%%%%%%%%%%%%%%%%%%%%%%%%%%%%%%%%%%%%%%%%%%

\photo{Cheese Crisp with Potatoes}

\newpage	%270 %%%%%%%%%%%%%%%%%%%%%%%%%%%%%%%%%%%%%%%%%%%%%%%%%%%%%%%%%%%%%%%%%%

\recipe{Milk Fritters}
\italianrecipe{Crema Fritta}
\region{Emillio-Romagna}
\ingredient{Milk}
\ingredient{Eggs}

\recipe{Borage Fritters}
\italianrecipe{Frisceu di Erbe}
\region{Liguria}
\ingredient{Eggs}

\newpage	%271 %%%%%%%%%%%%%%%%%%%%%%%%%%%%%%%%%%%%%%%%%%%%%%%%%%%%%%%%%%%%%%%%%%

\recipe{Buckwheat and Cheese Fritters}
\italianrecipe{Sciatt}
\region{Lombardia}
\ingredient{Milk}
\ingredient{Buckwheat Flour}
\ingredient{Rice Flour}
\ingredient{Cheese, Brie}	% Nearest thing to Scimud I could find

\recipe{Pan-Fried Zucchini Croquettes}
\italianrecipe{Polpette di Zucchine}
\region{Puglia}
\ingredient{Zucchini}
\ingredient{Eggs}

\newpage	%272 %%%%%%%%%%%%%%%%%%%%%%%%%%%%%%%%%%%%%%%%%%%%%%%%%%%%%%%%%%%%%%%%%%

\recipe{Fried Chicory}
\italianrecipe{Cicoria Impazzita}
\vegan
\region{Abruzzo}
\ingredient{Chicory}

\recipe{Pan-Fried Chicory Croquettes}
\italianrecipe{Polpette di Cicoria}
\region{Puglia}
\ingredient{Chicory}
\ingredient{Eggs}

\newpage	%273 %%%%%%%%%%%%%%%%%%%%%%%%%%%%%%%%%%%%%%%%%%%%%%%%%%%%%%%%%%%%%%%%%%

\recipe{Fried Dumplings Stuffed with Greens, Squash and Rice}
\italianrecipe{Barbagiuai alle Erbe}
\region{Liguria}
\ingredient{Chard}
\ingredient{Squash}
\ingredient{Rice}
\ingredient{Chard}

\newpage	%274 %%%%%%%%%%%%%%%%%%%%%%%%%%%%%%%%%%%%%%%%%%%%%%%%%%%%%%%%%%%%%%%%%%

\recipe{Pan-Fried Eggplant Croquettes}
\italianrecipe{Polpette di Melanzane}
\region{Calabria}
\ingredient{Aubergines}

\recipe{Eggplant Medallions}
\italianrecipe{Medaglioni di Melanzane}
\region{Friuli Venezia Giulia}
\ingredient{Aubergines}
\ingredient{Eggs}

\newpage	%275 %%%%%%%%%%%%%%%%%%%%%%%%%%%%%%%%%%%%%%%%%%%%%%%%%%%%%%%%%%%%%%%%%%

\photo{Pan-Fried Eggplant Croquettes}

\newpage	%276 %%%%%%%%%%%%%%%%%%%%%%%%%%%%%%%%%%%%%%%%%%%%%%%%%%%%%%%%%%%%%%%%%%

% What a mouthfull!
\recipe{Taleggio and Eggplant Fritters with Jerusalem Artichoke Puree}
\italianrecipe{Gnocco Fritto di Melanzane e Taleggio con Crema di Topinambur}
\region{Marche}
\ingredient{Aubergines}
\ingredient{Jerusalem Artichokes}
\ingredient{Cheese, Taleggio}
\ingredient{Eggs}

\newpage	%277 %%%%%%%%%%%%%%%%%%%%%%%%%%%%%%%%%%%%%%%%%%%%%%%%%%%%%%%%%%%%%%%%%%

\recipe{Fried Stuffed Artichokes in Tomato Sauce}
\italianrecipe{Carciofi Ripieni al Sugo}
\region{Campania}
\ingredient{Artichokes, Globe}
\ingredient{Tomatoes}

\newpage	%278 %%%%%%%%%%%%%%%%%%%%%%%%%%%%%%%%%%%%%%%%%%%%%%%%%%%%%%%%%%%%%%%%%%

\recipe{Fried Mushrooms}
\italianrecipe{Funghi fritti all'olio di Lentischo}
\region{Sardegna}
\ingredient{Mushrooms}

\newpage	%279 %%%%%%%%%%%%%%%%%%%%%%%%%%%%%%%%%%%%%%%%%%%%%%%%%%%%%%%%%%%%%%%%%%

\null	%force an empty page
% Unknown photo? Looks like fried bread - not fried Mushrooms!
%\photo{???}

\newpage	%280 %%%%%%%%%%%%%%%%%%%%%%%%%%%%%%%%%%%%%%%%%%%%%%%%%%%%%%%%%%%%%%%%%%

\recipe{Pan-Fried Wild Fennel Croquettes}
\italianrecipe{Polpettine di Finocchietto}
\region{Sicilia}
\ingredient{Fennel, Bulb}
\ingredient{Eggs}
\ingredient{Cheese, Tumazzo}

\recipe{Fried Milk Thistle}
\italianrecipe{Munaceddi Fritti}
\vegan
\region{Sicilia}
\ingredient{Milk Thistle}

\newpage	%281 %%%%%%%%%%%%%%%%%%%%%%%%%%%%%%%%%%%%%%%%%%%%%%%%%%%%%%%%%%%%%%%%%%

\recipe{Fried Seaweed}
\italianrecipe{Frittelle di Alghe Marine}
\region{Calabria}
\ingredient{Sea Lettuce}
% Because without this it is notionally hard to find the recipe
\ingredient{Seaweed}

\recipe{Squash Croquettes}
\italianrecipe{Fritelle di Zucca}
\region{Calabria}
\ingredient{Squash}
\ingredient{Eggs}

\newpage	%282 %%%%%%%%%%%%%%%%%%%%%%%%%%%%%%%%%%%%%%%%%%%%%%%%%%%%%%%%%%%%%%%%%%

\recipe{Cauliflower Fritters}
\italianrecipe{Frittelle di Sant'Antonio}
\region{Lazio}
\vegan
\ingredient{Cauliflower}

\newpage	%283 %%%%%%%%%%%%%%%%%%%%%%%%%%%%%%%%%%%%%%%%%%%%%%%%%%%%%%%%%%%%%%%%%%

\recipe{Pan-Fried Cabbage Bundles}
\italianrecipe{Polpette di Verza}
\region{Lombardia}
\ingredient{Cabbage, Savoy}
% There is a hidden braised cabbage recipe here!
\refcurrentrecipe{Saut\'{e}ed Cabbage in Tomato Sauce}

\newpage	%284 %%%%%%%%%%%%%%%%%%%%%%%%%%%%%%%%%%%%%%%%%%%%%%%%%%%%%%%%%%%%%%%%%%

\recipe{Chickpea Flour Fritters}
\italianrecipe{Panelle}
\region{Sicilia}
\vegan
\ingredient{Chickpea Flour}

\newpage	%285 %%%%%%%%%%%%%%%%%%%%%%%%%%%%%%%%%%%%%%%%%%%%%%%%%%%%%%%%%%%%%%%%%%

\photo{Chickpea Flour Fritters}

\newpage	%286 %%%%%%%%%%%%%%%%%%%%%%%%%%%%%%%%%%%%%%%%%%%%%%%%%%%%%%%%%%%%%%%%%%

\recipe{Squash Blossoms with Bean Stuffing}
\italianrecipe{Fiori di Trombetta Ripieni}
\region{Liguria}
\ingredient{Potatoes}
\ingredient{Cannellini Beans}
\ingredient{Squash Blossoms}

\newpage	%287 %%%%%%%%%%%%%%%%%%%%%%%%%%%%%%%%%%%%%%%%%%%%%%%%%%%%%%%%%%%%%%%%%%

\recipe{Stuffed Squash Blossoms}
\italianrecipe{Caponet}
\region{Piemonte}
\ingredient{Zucchini}
\ingredient{Squash Blossoms}

%%%%%%%%%%%%%%%%%%%%%%%%%%%%% END OF PANCAKES %%%%%%%%%%%%%%%%%%%%%%%%%%%%%%%%%%%%%

%%%%%%%%%%%%%%%%%%%%%%%%%%%%% START OF PIZZAS %%%%%%%%%%%%%%%%%%%%%%%%%%%%%%%%%%%%%

% Technically 'Pizzas and Savory Tarts'

% First recipe is on page 290 - set to 289, and then bump to new page

\setcounter{page}{289} %289 %%%%%%%%%%%%%%%%%%%%%%%%%%%%%%%%%%%%%%%%%%%%%%%%%%%%%%%%%%%%%%%%%%

\newpage	%290 %%%%%%%%%%%%%%%%%%%%%%%%%%%%%%%%%%%%%%%%%%%%%%%%%%%%%%%%%%%%%%%%%%
\category{Pizzas and Savory Tarts}

\recipe{Vegan Pizza}
\italianrecipe{Pizza Vegana}
\vegan
\region{Lombardia}
% This is made from Oat, Rice and Cornmean - so might also be gluten free?
% No real critical ingredients though? Mix of veg

\newpage	%291 %%%%%%%%%%%%%%%%%%%%%%%%%%%%%%%%%%%%%%%%%%%%%%%%%%%%%%%%%%%%%%%%%%

\photo{Vegan Pizza}

\newpage	%292 %%%%%%%%%%%%%%%%%%%%%%%%%%%%%%%%%%%%%%%%%%%%%%%%%%%%%%%%%%%%%%%%%%

\recipe{Stuffed Cornmeal Crust Pizza}
\italianrecipe{Pizza de Turco con Cicoria}
\vegan
\region{Lazio}
\ingredient{Chicory}

\newpage	%293 %%%%%%%%%%%%%%%%%%%%%%%%%%%%%%%%%%%%%%%%%%%%%%%%%%%%%%%%%%%%%%%%%%

\recipe{Flatbread with Sheep's Cheese}
\italianrecipe{Ciaccia al Formaggio}
\region{Umbria}
\ingredient{Cheese, Sheep's}

\newpage	%294 %%%%%%%%%%%%%%%%%%%%%%%%%%%%%%%%%%%%%%%%%%%%%%%%%%%%%%%%%%%%%%%%%%

\recipe{Focaccia with Mixed Vegetables}
\italianrecipe{Focaccia Ortolana}
\region{March}
\ingredient{Asparagus}
\ingredient{Artichokes, Globe}
\ingredient{Potatoes}
\ingredient{Wild Greens}

\newpage	%295 %%%%%%%%%%%%%%%%%%%%%%%%%%%%%%%%%%%%%%%%%%%%%%%%%%%%%%%%%%%%%%%%%%

\recipe{Focaccia Stuffed with Onions}
\italianrecipe{Fucazza Chiena cu la Cipodda}
\vegan
\region{Puglia}
\ingredient{Onions, Red}
\ingredient{Tomatoes}
\ingredient{Potatoes}

\newpage	%296 %%%%%%%%%%%%%%%%%%%%%%%%%%%%%%%%%%%%%%%%%%%%%%%%%%%%%%%%%%%%%%%%%%

\recipe{Focaccia Stuffed with Greens}
\italianrecipe{Focaccia Farcita alle Erbe}
\region{Marche}
\ingredient{Potatoes}
\ingredient{Wild Greens}

\newpage	%297 %%%%%%%%%%%%%%%%%%%%%%%%%%%%%%%%%%%%%%%%%%%%%%%%%%%%%%%%%%%%%%%%%%

\photo{Focaccia Stuffed with Greens}

\newpage	%298 %%%%%%%%%%%%%%%%%%%%%%%%%%%%%%%%%%%%%%%%%%%%%%%%%%%%%%%%%%%%%%%%%%

\recipe{Fried Small Pizzas and Turnovers}
\italianrecipe{Pizzelle e Panzerotti}
\region{Campania}
% No real key ingredients

\newpage	%299 %%%%%%%%%%%%%%%%%%%%%%%%%%%%%%%%%%%%%%%%%%%%%%%%%%%%%%%%%%%%%%%%%%

\photo{Fried Small Pizzas and Turnovers}

\newpage	%300 %%%%%%%%%%%%%%%%%%%%%%%%%%%%%%%%%%%%%%%%%%%%%%%%%%%%%%%%%%%%%%%%%%

\recipe{Fried Dough with Spicy Tomatoes}
\italianrecipe{Arvolto con Pomodori Piccanti}
\vegan
\region{Umbria}
\ingredient{Tomatoes}

\newpage	%301 %%%%%%%%%%%%%%%%%%%%%%%%%%%%%%%%%%%%%%%%%%%%%%%%%%%%%%%%%%%%%%%%%%

\recipe{Savory Pie with Greens and Rice}
\italianrecipe{Torta Verde}
\region{Piemonte}
\ingredient{Spinach}
\ingredient{Chard}
\ingredient{Nettles}
\ingredient{Rice}
\ingredient{Eggs}

\newpage	%302 %%%%%%%%%%%%%%%%%%%%%%%%%%%%%%%%%%%%%%%%%%%%%%%%%%%%%%%%%%%%%%%%%%

\recipe{Graffiti Eggplant Tart}
\italianrecipe{Torta di Melanzane Striate}
\vegan
\region{Lombardia}
\ingredient{Aubergines}

\newpage	%303 %%%%%%%%%%%%%%%%%%%%%%%%%%%%%%%%%%%%%%%%%%%%%%%%%%%%%%%%%%%%%%%%%%

\photo{Graffiti Eggplant Tart}

\newpage	%304 %%%%%%%%%%%%%%%%%%%%%%%%%%%%%%%%%%%%%%%%%%%%%%%%%%%%%%%%%%%%%%%%%%

\recipe{Trombetta Squash Tart}
\italianrecipe{Torta di Trombette}
\region{Liguria}
\ingredient{Squash, Trombetta}

\newpage	%305 %%%%%%%%%%%%%%%%%%%%%%%%%%%%%%%%%%%%%%%%%%%%%%%%%%%%%%%%%%%%%%%%%%

\recipe{Artichoke Pie}
\italianrecipe{Panada di Carciofi}
\region{Sardegna}
\vegan
\ingredient{Artichokes, Globe}

\newpage	%306 %%%%%%%%%%%%%%%%%%%%%%%%%%%%%%%%%%%%%%%%%%%%%%%%%%%%%%%%%%%%%%%%%%

\recipe{Easter Pie with Greens and Herbs}
\italianrecipe{Torta Pasqualina}
\region{Liguria}
\ingredient{Chard}
\ingredient{Cheese, Ricotta}

\newpage	%307 %%%%%%%%%%%%%%%%%%%%%%%%%%%%%%%%%%%%%%%%%%%%%%%%%%%%%%%%%%%%%%%%%%

\photo{Easter Pie with Greens and Herbs}

\newpage	%308 %%%%%%%%%%%%%%%%%%%%%%%%%%%%%%%%%%%%%%%%%%%%%%%%%%%%%%%%%%%%%%%%%%

\recipe{Pie with Escarole, Olives and Raisins}
\italianrecipe{Pizza di Scarola}
\region{Campania}
\ingredient{Endive}	% Escarole in the recipe
\ingredient{Olives}
\ingredient{Raisins}

\newpage	%309 %%%%%%%%%%%%%%%%%%%%%%%%%%%%%%%%%%%%%%%%%%%%%%%%%%%%%%%%%%%%%%%%%%

\recipe{Sheep's Cheese and Saffron Tart}
\italianrecipe{Casadinas}
\region{Sardegna}
\ingredient{Cheese, Sheep's}
\ingredient{Saffron}

\newpage	%310 %%%%%%%%%%%%%%%%%%%%%%%%%%%%%%%%%%%%%%%%%%%%%%%%%%%%%%%%%%%%%%%%%%

\recipe{Flaky Pie with Greens}
\italianrecipe{Torta con Verdure}
\vegan
\region{Umbria}
\ingredient{Spinach}
\ingredient{Chard}
\ingredient{Chicory}

\newpage	%311 %%%%%%%%%%%%%%%%%%%%%%%%%%%%%%%%%%%%%%%%%%%%%%%%%%%%%%%%%%%%%%%%%%

\recipe{Cauliflower Tart}
\italianrecipe{Broccolo in Crosta}
\region{Trentino}
\ingredient{Cauliflower}

\newpage	%312 %%%%%%%%%%%%%%%%%%%%%%%%%%%%%%%%%%%%%%%%%%%%%%%%%%%%%%%%%%%%%%%%%%

\recipe{Malga Cheese and Greens Tartlets}
\italianrecipe{Tortino di Formaggio di Malga ed Erbette}
\region{Alto Adige}
\ingredient{Cheese, Malga}

\newpage	%313 %%%%%%%%%%%%%%%%%%%%%%%%%%%%%%%%%%%%%%%%%%%%%%%%%%%%%%%%%%%%%%%%%%

\photo{Malga Cheese and Greens Tartlets}

\newpage	%314 %%%%%%%%%%%%%%%%%%%%%%%%%%%%%%%%%%%%%%%%%%%%%%%%%%%%%%%%%%%%%%%%%%

\recipe{Potato and Mushroom Tart}
\italianrecipe{Baciocca con i Funghi}
\vegan
\region{Liguria}
\ingredient{Potatoes}
\ingredient{Mushrooms}

\newpage	%315 %%%%%%%%%%%%%%%%%%%%%%%%%%%%%%%%%%%%%%%%%%%%%%%%%%%%%%%%%%%%%%%%%%

\recipe{Spinach Roll}
\italianrecipe{Rotolo di Spinaci}
\region{Alto Adige}
\ingredient{Potatoes}
\ingredient{Spinach}

\newpage	%316 %%%%%%%%%%%%%%%%%%%%%%%%%%%%%%%%%%%%%%%%%%%%%%%%%%%%%%%%%%%%%%%%%%

\recipe{Cipollini Onion Tart in a Kamut Crust}
\italianrecipe{Crostata di Cipolline}
\region{Marche}
\ingredient{Onions, Caipollini}

\newpage	%317 %%%%%%%%%%%%%%%%%%%%%%%%%%%%%%%%%%%%%%%%%%%%%%%%%%%%%%%%%%%%%%%%%%

\recipestart{Semolina Pies with Fresh Tomatoes}
\italianrecipe{Coccoi Prena}
\vegan
\region{Sardegna}
\ingredient{Tomatoes}

\newpage	%318 %%%%%%%%%%%%%%%%%%%%%%%%%%%%%%%%%%%%%%%%%%%%%%%%%%%%%%%%%%%%%%%%%%

\recipestop{Semolina Pies with Fresh Tomatoes}

\recipe{Vegetable Strudel}
\italianrecipe{Strudel di Verdure}
\region{Lombardia}
% A hidden shortcrust pastry recipe...
\refcurrentrecipe{Short-Crust Pastry}
% Too many sorts of vegetable to list. Nothing individually critical?

%%%%%%%%%%%%%%%%%%%%%%%%%%%%% END OF PIZZAS %%%%%%%%%%%%%%%%%%%%%%%%%%%%%%%%%%%%%

%%%%%%%%%%%%%%%%%%%%%%%%%%%%% START OF CASSEROLES %%%%%%%%%%%%%%%%%%%%%%%%%%%%%%%%%%%%%

% Technically, 'Casseroles and other Baked Dishes'

% First recipe is on page 322 - set to 321, and then bump to new page

\setcounter{page}{321} %321 %%%%%%%%%%%%%%%%%%%%%%%%%%%%%%%%%%%%%%%%%%%%%%%%%%%%%%%%%%%%%%%%%%

\newpage	%322 %%%%%%%%%%%%%%%%%%%%%%%%%%%%%%%%%%%%%%%%%%%%%%%%%%%%%%%%%%%%%%%%%%

\category{Casseroles}

\recipe{Vegetable Cake}
\italianrecipe{Gatt\'{o} di Verdure}
\region{Lombardia}
\ingredient{Cheese, Taleggio}

\recipe{Escarole and Mozzarella Cakes}
\italianrecipe{Impasticciata}
\region{Umbria}
\ingredient{Endive}
\ingredient{Cheese, Mozzarella}

\newpage	%323 %%%%%%%%%%%%%%%%%%%%%%%%%%%%%%%%%%%%%%%%%%%%%%%%%%%%%%%%%%%%%%%%%%

% Sounds very much like a Rosti...
\recipe{Baked Potato Cake}
\italianrecipe{Tortel di Patate}
\vegan
\region{Tentino}
\ingredient{Potatoes}

\newpage	%324 %%%%%%%%%%%%%%%%%%%%%%%%%%%%%%%%%%%%%%%%%%%%%%%%%%%%%%%%%%%%%%%%%%

\recipe{Yeast-Risen Potato Cake}
\italianrecipe{Pinza di Patate}
\vegan
\region{Tentino}
\ingredient{Potatoes}

\newpage	%325 %%%%%%%%%%%%%%%%%%%%%%%%%%%%%%%%%%%%%%%%%%%%%%%%%%%%%%%%%%%%%%%%%%

\recipe{Green Bean, Carrot and Potato Loaf}
\italianrecipe{Polpettone di Fagiolini Carote e Patate}
\region{Liguria}
\ingredient{Beans, Green}
\ingredient{Carrots}
\ingredient{Potatoes}

\newpage	%326 %%%%%%%%%%%%%%%%%%%%%%%%%%%%%%%%%%%%%%%%%%%%%%%%%%%%%%%%%%%%%%%%%%

\recipe{Eggplant Parmigiana with Tomato Sauce}
\italianrecipe{Parmigiana di Melanzane}
\region{Campania}
\ingredient{Aubergines}
\ingredient{Tomatoes}
\ingredient{Cheese, Mozzarella}

\newpage	%327 %%%%%%%%%%%%%%%%%%%%%%%%%%%%%%%%%%%%%%%%%%%%%%%%%%%%%%%%%%%%%%%%%%

\photo{Eggplant Parmigiana with Tomato Sauce}

\newpage	%328 %%%%%%%%%%%%%%%%%%%%%%%%%%%%%%%%%%%%%%%%%%%%%%%%%%%%%%%%%%%%%%%%%%

\recipe{White Eggplant Parmigiana}
\italianrecipe{Parmigiana Bianca di Melanzane}
\region{Puglia}
\ingredient{Aubergines}
\ingredient{Cheese, Mozzarella}

\newpage	%329 %%%%%%%%%%%%%%%%%%%%%%%%%%%%%%%%%%%%%%%%%%%%%%%%%%%%%%%%%%%%%%%%%%

\recipe{Zucchini Parmigiana}
\italianrecipe{Parmigiana di Zucchine}
\region{Calabria}
\ingredient{Zucchini}
\ingredient{Cheese, Provolone}

\newpage	%330 %%%%%%%%%%%%%%%%%%%%%%%%%%%%%%%%%%%%%%%%%%%%%%%%%%%%%%%%%%%%%%%%%%

\recipe{Friarelli Pepper and Potato Timbale}
\italianrecipe{Timballeto di Friarelli e Patate}
\region{Campania}
\ingredient{Peppers, Green}
\ingredient{Potatoes}
\ingredient{Cheese, Ricotta} %Because who can find Cacioricotta Cilentano?

\recipe{Layered Roasted Potatoes}
\italianrecipe{Patate al Forno}
\region{Basilicata}
\ingredient{Potatoes}

\newpage	%331 %%%%%%%%%%%%%%%%%%%%%%%%%%%%%%%%%%%%%%%%%%%%%%%%%%%%%%%%%%%%%%%%%%

\recipe{Porcini and Potato Casserole}
\italianrecipe{Pasticcio di Patate e Porcini}
\vegan
\region{Lombardia}
\ingredient{Potatoes}
\ingredient{Mushrooms, Porcini}

\newpage	%332 %%%%%%%%%%%%%%%%%%%%%%%%%%%%%%%%%%%%%%%%%%%%%%%%%%%%%%%%%%%%%%%%%%

\recipe{Buckwheat, Potatoes and Green Beans Casserole}
\italianrecipealtsort{Ma\"{u}sc}{Mausc}
\region{Lombardia}
\ingredient{Buckwheat Flour}
\ingredient{Potatoes}
\ingredient{Beans, Green}

\recipe{Baked Leeks with Fried Eggs}
\italianrecipe{Porri Gratinati e Uova al Tegamino}
\region{Lombardia}
\ingredient{Leeks}
\ingredient{Eggs}

\newpage	%333 %%%%%%%%%%%%%%%%%%%%%%%%%%%%%%%%%%%%%%%%%%%%%%%%%%%%%%%%%%%%%%%%%%

\photo{Buckwheat, Potatoes and Green Beans Casserole}

\newpage	%334 %%%%%%%%%%%%%%%%%%%%%%%%%%%%%%%%%%%%%%%%%%%%%%%%%%%%%%%%%%%%%%%%%%

\recipe{Baked Mushrooms with Potatoes}
\italianrecipe{Funghi al Forno con Patate}
\region{Trentino}
\ingredient{Potatoes}
\ingredient{Mushrooms}

\newpage	%335 %%%%%%%%%%%%%%%%%%%%%%%%%%%%%%%%%%%%%%%%%%%%%%%%%%%%%%%%%%%%%%%%%%

\recipe{Potatoes with Melted Cheese}
\italianrecipe{Patate Gratinate}
\region{Valle d'Aosta}
\ingredient{Potatoes}
\ingredient{Cheese, Fontina}

\recipe{Roasted Tomatoes}
\italianrecipe{Pomodori Arrosto}
\vegan
\region{Marche}
\ingredient{Tomatoes}

\newpage	%336 %%%%%%%%%%%%%%%%%%%%%%%%%%%%%%%%%%%%%%%%%%%%%%%%%%%%%%%%%%%%%%%%%%

\recipe{Baked Squash Blossoms with Cacioricotta Cheese}
\italianrecipe{Fiori di Zucchina al Cacioricotta}
\region{Puglia}
\ingredient{Squash Blossoms}
\ingredient{Cheese, Ricotta} % Cacioricotta

\newpage	%337 %%%%%%%%%%%%%%%%%%%%%%%%%%%%%%%%%%%%%%%%%%%%%%%%%%%%%%%%%%%%%%%%%%

\recipe{Stuffed Peppers}
\italianrecipe{Friggitelli alla Verza}
\region{Lazio}
\ingredient{Peppers, Green}
\ingredient{Cheese, Ricotta} % Cacioricotta

\newpage	%338 %%%%%%%%%%%%%%%%%%%%%%%%%%%%%%%%%%%%%%%%%%%%%%%%%%%%%%%%%%%%%%%%%%

\recipe{Eggplant Cups}
\italianrecipe{Cialde di Melanzane}
\region{Piemonte}
\ingredient{Aubergines}

\recipe{Roast Mixed Vegetables}
\italianrecipe{Zabbinata}
\vegan
\region{Sicilia}
\ingredient{Potatoes}
\ingredient{Zucchini}
\ingredient{Peppers, Bell}
\ingredient{Tomatoes}

\newpage	%339 %%%%%%%%%%%%%%%%%%%%%%%%%%%%%%%%%%%%%%%%%%%%%%%%%%%%%%%%%%%%%%%%%%

\photo{Eggplant Cups}

\newpage	%340 %%%%%%%%%%%%%%%%%%%%%%%%%%%%%%%%%%%%%%%%%%%%%%%%%%%%%%%%%%%%%%%%%%

\recipe{Stuffed Vegetables}
\italianrecipe{Verdure Ripiene}
\region{Liguria}
% Just pick a couple of the veg that I am likely to be looking at in the index
\ingredient{Zucchini}
\ingredient{Aubergines}

\newpage	%341 %%%%%%%%%%%%%%%%%%%%%%%%%%%%%%%%%%%%%%%%%%%%%%%%%%%%%%%%%%%%%%%%%%

\recipe{Round Eggplant with White Bean and Porcini Stuffing}
\italianrecipe{Melanzane Rosse di Rotonda Ripiene}
\region{Basilicata}
\ingredient{Aubergines}
\ingredient{Beans, White}	% Poverello - or whatever nearest equivalent we have

\newpage	%342 %%%%%%%%%%%%%%%%%%%%%%%%%%%%%%%%%%%%%%%%%%%%%%%%%%%%%%%%%%%%%%%%%%

\recipe{Stuffed Eggplant}
\italianrecipe{Melanzane Ripiene}
\region{Puglia}
\ingredient{Aubergines}
\ingredient{Beans, White}	% Poverello - or whatever nearest equivalent we have

\newpage	%343 %%%%%%%%%%%%%%%%%%%%%%%%%%%%%%%%%%%%%%%%%%%%%%%%%%%%%%%%%%%%%%%%%%

\photo{Stuffed Eggplant}

\newpage	%344 %%%%%%%%%%%%%%%%%%%%%%%%%%%%%%%%%%%%%%%%%%%%%%%%%%%%%%%%%%%%%%%%%%

\recipe{Red Onions with Eggplant Stuffing}
\italianrecipe{Cipolle Rosse Ripiene}
\region{Calabria}
\ingredient{Onions, Red}
\ingredient{Aubergines}

\recipe{Roasted Squash}
\italianrecipe{Zucca al Forno}
\region{Lombardia}
\ingredient{Squash}
\ingredient{Aubergines}

\newpage	%345 %%%%%%%%%%%%%%%%%%%%%%%%%%%%%%%%%%%%%%%%%%%%%%%%%%%%%%%%%%%%%%%%%%

\photo{Red Onions with Eggplant Stuffing}

\newpage	%346 %%%%%%%%%%%%%%%%%%%%%%%%%%%%%%%%%%%%%%%%%%%%%%%%%%%%%%%%%%%%%%%%%%

\recipe{Baked Spinach Cr\^{e}pe Pinwheels}
\italianrecipe{Girelle di Crespella Gratinate}
\region{Toscana}
\ingredient{Spinach}
% There is a hidden Crepe recipe here...
\refcurrentrecipealtsort{Cr\^{e}pe}{Crepe}

\newpage	%347 %%%%%%%%%%%%%%%%%%%%%%%%%%%%%%%%%%%%%%%%%%%%%%%%%%%%%%%%%%%%%%%%%%

\recipe{Zucchini and Squash Blossoms in Tomato Sauce}
\italianrecipe{Zucchine e Fiori di Zucca in Salsa di Pomodoro}
\region{Piemonte}
\ingredient{Squash Blossoms}
\ingredient{Tomatoes}

%%%%%%%%%%%%%%%%%%%%%%%%%%%%% END OF CASSEROLES %%%%%%%%%%%%%%%%%%%%%%%%%%%%%%%%%%%%%

%%%%%%%%%%%%%%%%%%%%%%%%%%%%% START OF SAUTES %%%%%%%%%%%%%%%%%%%%%%%%%%%%%%%%%%%%%

% Technically 'Sautes, Braises and Grilled Vegetables'
% First recipe is on page 350

\setcounter{page}{349} %349 %%%%%%%%%%%%%%%%%%%%%%%%%%%%%%%%%%%%%%%%%%%%%%%%%%%%%%%%%%%%%%%%%%

\newpage	%350 %%%%%%%%%%%%%%%%%%%%%%%%%%%%%%%%%%%%%%%%%%%%%%%%%%%%%%%%%%%%%%%%%%

% Watch out for index sort order - seems OK so far...
\category{Saut\'{e}s}

\recipe{Saut\'{e}ed Potatoes}
\italianrecipe{Patate in Padella}
\region{Valle d'Aosta}
\ingredient{Potatoes}
\ingredient{Leeks}

\recipe{Potato and Pepper Saut\'{e}e}
\italianrecipe{Pipi e Patate}
\vegan
\region{Calabria}
\ingredient{Potatoes}
\ingredient{Peppers, Bell}

\newpage	%351 %%%%%%%%%%%%%%%%%%%%%%%%%%%%%%%%%%%%%%%%%%%%%%%%%%%%%%%%%%%%%%%%%%

\recipe{Crushed Potatoes with Onion and Cheese}
\italianrecipe{Patugol}
\region{Trentino}
\ingredient{Potatoes}
\ingredient{Leeks}

% Yay - Rosti! The actual recipe description is very hard to follow - seems broken
\recipe{Stovetop Potato Cake}
\italianrecipealtsort{R\"{o}sti}{Rosti}
\region{Lombardia}
\ingredient{Potatoes}

\newpage	%352 %%%%%%%%%%%%%%%%%%%%%%%%%%%%%%%%%%%%%%%%%%%%%%%%%%%%%%%%%%%%%%%%%%

% Err, the title says Peppers, but there are no peppers in the recipe!!!
% There are however Onions and Celery
\recipe{Saut\'{e}ed Eggplant and Peppers with Capers, Olives and Zagara Honey}
\italianrecipe{Caponata al Miele di Zagara}
\region{Sicilia}
\ingredient{Aubergines}
\ingredient{Onions}
\ingredient{Celery}

\newpage	%353 %%%%%%%%%%%%%%%%%%%%%%%%%%%%%%%%%%%%%%%%%%%%%%%%%%%%%%%%%%%%%%%%%%

\photo{Saut\'{e}ed Eggplant and Peppers with Capers, Olives and Zagara Honey}

\newpage	%354 %%%%%%%%%%%%%%%%%%%%%%%%%%%%%%%%%%%%%%%%%%%%%%%%%%%%%%%%%%%%%%%%%%

% Ratatouille !
\recipe{Sweet and Sour Saut\'{e}ed Vegetables}
\italianrecipe{Ratatoia Agrodolce}
\vegan
\region{Sicilia}
\refcurrentrecipe{Ratatouille}
\ingredient{Peppers, Bell}
\ingredient{Zucchini}

\recipe{Garlicky Soft Eggplant}
\vegan
\region{Veneto}
\ingredient{Aubergines}

\newpage	%355 %%%%%%%%%%%%%%%%%%%%%%%%%%%%%%%%%%%%%%%%%%%%%%%%%%%%%%%%%%%%%%%%%%

\recipe{Bell Pepper Saut\'{e} with Olivers and Capers}
\italianrecipe{Peperoni con Olive e Capperi}
\vegan	% Even though not denoted so in the book?
\region{Campania}
\ingredient{Peppers, Bell}
\ingredient{Olives}
\ingredient{Capers}

\recipe{Green Pepper Saut\'{e}}
\italianrecipe{Peperoni Verdi al Tegame}
\vegan
\region{Umbria}
\ingredient{Peppers, Bell}
\ingredient{Tomatoes}

\newpage	%356 %%%%%%%%%%%%%%%%%%%%%%%%%%%%%%%%%%%%%%%%%%%%%%%%%%%%%%%%%%%%%%%%%%

\recipe{Saut\'{e}ed Bell Peppers}
\italianrecipe{Peperonata}
\vegan
\region{Lombardia}
\ingredient{Peppers, Bell}
\ingredient{Aubergines}

\recipe{Saut\'{e}ed Green Beans and Potatoes}
\italianrecipe{Patao}
\region{Lombardia}
\ingredient{Beans, Green}
\ingredient{Potatoes}

\newpage	%357 %%%%%%%%%%%%%%%%%%%%%%%%%%%%%%%%%%%%%%%%%%%%%%%%%%%%%%%%%%%%%%%%%%

\recipe{Green Beans Saut\'{e}ed with Tomatoes}
\italianrecipe{Fagiolini al Pomodoro}
\vegan
\region{Toscana}
\ingredient{Beans, Green}
\ingredient{Tomatoes}

\recipe{Green Beans with Garlic and Olive Oil}
\italianrecipe{Cornetti in Salsa}
\vegan
\region{Veneto}
\ingredient{Beans, Green}
\ingredient{Tomatoes}

\newpage	%358 %%%%%%%%%%%%%%%%%%%%%%%%%%%%%%%%%%%%%%%%%%%%%%%%%%%%%%%%%%%%%%%%%%

\recipe{Potatoes, Greens and Breadcrumbs}
\italianrecipe{Verdure Ammollicate}
\region{Calabria}
\ingredient{Chard}
\ingredient{Chicory}
\ingredient{Potatoes}

\recipe{Saut\'{e}ed Mushrooms}
\italianrecipe{Funghi Trifolati}
\vegan
\region{Toscana}
\ingredient{Mushrooms}

\newpage	%359 %%%%%%%%%%%%%%%%%%%%%%%%%%%%%%%%%%%%%%%%%%%%%%%%%%%%%%%%%%%%%%%%%%

\recipe{Preserved Stewed Vegetables}
\italianrecipe{Repouta}
\vegan
\region{Valle d'Aosta}
% Do we list all the veg or not - let's start by listing them...
\ingredient{Cabbage, Savoy}
\ingredient{Chard}
\ingredient{Turnips}
\ingredient{Beetroot}

\newpage	%360 %%%%%%%%%%%%%%%%%%%%%%%%%%%%%%%%%%%%%%%%%%%%%%%%%%%%%%%%%%%%%%%%%%

\recipe{Stewed Eggplant, Tomatoes and Bell Peppers}
\italianrecipe{Verdure Saporite}
\vegan
\region{Basilicata}
\ingredient{Aubergines}
\ingredient{Tomatoes}
\ingredient{Peppers, Bell}

\recipe{Eggplant Stewed with Peppers, Potatoes and Zucchini}
\italianrecipe{Ciaudedda}
\vegan
\region{Basilicata}
\ingredient{Aubergines}
\ingredient{Peppers, Bell}
\ingredient{Potatoes}
\ingredient{Zucchini}

\newpage	%361 %%%%%%%%%%%%%%%%%%%%%%%%%%%%%%%%%%%%%%%%%%%%%%%%%%%%%%%%%%%%%%%%%%

\recipe{Stewed Escarole}
\italianrecipe{Scarola Sufata}
\vegan
\region{Campania}
\ingredient{Chicory}	% Escarole
\ingredient{Tomatoes}

\recipe{Smothered Cabbage}
\italianrecipe{Verze Sofegae}
\region{Veneto}
\ingredient{Cabbage, Savoy}

\newpage	%362 %%%%%%%%%%%%%%%%%%%%%%%%%%%%%%%%%%%%%%%%%%%%%%%%%%%%%%%%%%%%%%%%%%

\recipe{Stewed Sauerkraut}
\italianrecipe{Crauti Stufati}
\vegan
\region{Trentino}
\ingredient{Sauerkraut}

\recipe{Braised Kohlrabi}
\italianrecipe{Cavolo Rapa Soffocato}
\region{Alto Adige}
\ingredient{Kohlrabi}

\newpage	%363 %%%%%%%%%%%%%%%%%%%%%%%%%%%%%%%%%%%%%%%%%%%%%%%%%%%%%%%%%%%%%%%%%%

\recipe{Braised Potatoes and Artichokes}
\italianrecipe{Cianfotta di Patae e Carciofi}
\vegan
\region{Campania}
\ingredient{Potatoes}
\ingredient{Artichokes, Globe}

\recipe{Braised Potatoes and Mushrooms}
\italianrecipe{Patate Silane 'mpacchiate al Funghi}
\vegan
\region{Calabria}
\ingredient{Potatoes}
\ingredient{Mushrooms, Porcini}

\newpage	%364 %%%%%%%%%%%%%%%%%%%%%%%%%%%%%%%%%%%%%%%%%%%%%%%%%%%%%%%%%%%%%%%%%%

\recipe{Braised Onions with Chervil}
\italianrecipe{Umido di Cipolline al Cerfoglio}
\vegan
\region{Emilia-Romagna}
\ingredient{Onions}
\ingredient{Artichokes, Globe}

\recipe{Glazed Shallots}
\italianrecipe{Scalogni Glassati}
% \vegan % Heh, and here are the anomolies in the book - the above recipe is listed
% as vegan, but has butter in it. This one is listed as not vegan, and also has butter in
% it - as well as honey? Maybe the Italian original had different ideas about what is Vegan,
% or maybe it is just mistakes?
\region{Emilia-Romagna}
\ingredient{Shallots}
\ingredient{Honey}

\newpage	%365 %%%%%%%%%%%%%%%%%%%%%%%%%%%%%%%%%%%%%%%%%%%%%%%%%%%%%%%%%%%%%%%%%%

\recipe{Carrots Simmered in Sweet and Sour Sauce}
\italianrecipe{Carote in Agrodolce}
\vegan
\region{Veneto}
\ingredient{Carrots}

\recipe{Cardoons in Tomato Sauce}
\italianrecipe{Gobbi Trippati}
\region{Toscana}
\ingredient{Cardoons}

\newpage	%366 %%%%%%%%%%%%%%%%%%%%%%%%%%%%%%%%%%%%%%%%%%%%%%%%%%%%%%%%%%%%%%%%%%

\recipe{Sugar Snap Peas Braised with Tomatoes and Basil}
\italianrecipe{Baggianata}
\vegan
\region{Lombardia}
\ingredient{Peas, Sugar Snap}
\ingredient{Tomatoes}

\recipe{Braised Baby Artichokes}
\italianrecipe{Castraure in Tecia}
\vegan
\region{Veneto}
\ingredient{Artichokes, Baby}

\newpage	%367 %%%%%%%%%%%%%%%%%%%%%%%%%%%%%%%%%%%%%%%%%%%%%%%%%%%%%%%%%%%%%%%%%%

\recipe{Slow-Cooked Stuffed Artichokes}
\italianrecipe{Carciofi alla Romana}
\vegan
\region{Lazio}
\ingredient{Artichokes, Globe}

\recipe{Braised Fennel}
\italianrecipe{Finocchi alla Giudia}
\vegan
\region{Lazio}
\ingredient{Fennel, Bulb}

\newpage	%368 %%%%%%%%%%%%%%%%%%%%%%%%%%%%%%%%%%%%%%%%%%%%%%%%%%%%%%%%%%%%%%%%%%

\recipe{Braised Chard Stems}
\italianrecipe{Coste di Bietole alla Veneziana}
\vegan
\region{Veneto}
\ingredient{Chard, Stems}

\recipe{Grilled Eggplant with Garlic and Capers}
\italianrecipe{Melanzane all'eoliana}
\vegan
\region{Sicilia}
\ingredient{Aubergines}

\newpage	%369 %%%%%%%%%%%%%%%%%%%%%%%%%%%%%%%%%%%%%%%%%%%%%%%%%%%%%%%%%%%%%%%%%%

\recipe{Grilled Mushrooms}
\italianrecipe{Cappelle di Morecci in Gratella}
\vegan
\region{Toscana}
\ingredient{Mushrooms, Porcini}

\newpage	%370 %%%%%%%%%%%%%%%%%%%%%%%%%%%%%%%%%%%%%%%%%%%%%%%%%%%%%%%%%%%%%%%%%%

\recipe{Stuffed Grape Leaves}
\italianrecipe{Capuss}
\region{Trentino}
\ingredient{Chard}
\ingredient{Grape Leaves}

\newpage	%371 %%%%%%%%%%%%%%%%%%%%%%%%%%%%%%%%%%%%%%%%%%%%%%%%%%%%%%%%%%%%%%%%%%

\photo{Stuffed Grape Leaves}

%%%%%%%%%%%%%%%%%%%%%%%%%%%%% END OF SAUTES %%%%%%%%%%%%%%%%%%%%%%%%%%%%%%%%%%%%%

%%%%%%%%%%%%%%%%%%%%%%%%%%%%% START OF DESSERTS %%%%%%%%%%%%%%%%%%%%%%%%%%%%%%%%%%%%%

% First recipe is on page 374

\setcounter{page}{373} %373 %%%%%%%%%%%%%%%%%%%%%%%%%%%%%%%%%%%%%%%%%%%%%%%%%%%%%%%%%%%%%%%%%%

\newpage	%374 %%%%%%%%%%%%%%%%%%%%%%%%%%%%%%%%%%%%%%%%%%%%%%%%%%%%%%%%%%%%%%%%%%

\category{Desserts}

\recipe{Chocolate Almond Caramel Pudding from Cogne}
\italianrecipe{Crema di Cogne}
\region{Valle d'Aosta}
\ingredient{Nuts, Almonds}
\ingredient{Gianduia}

\recipe{Almond Pudding}
\italianrecipe{Biancomangiare di Mandorle}
\vegan
\region{Scicilia}
\ingredient{Nuts, Almonds}

\newpage	%375 %%%%%%%%%%%%%%%%%%%%%%%%%%%%%%%%%%%%%%%%%%%%%%%%%%%%%%%%%%%%%%%%%%

\recipe{Chestnut Cream}
\italianrecipe{Crema di Castagne}
\vegan
\region{Basilicata}
\ingredient{Chestnuts}

\recipe{Bronte Pistachio Pudding}
\italianrecipe{Crema di Pistacchi di Bronte}
\region{Sicilia}
\ingredient{Nuts, Pistachios}

\newpage	%376 %%%%%%%%%%%%%%%%%%%%%%%%%%%%%%%%%%%%%%%%%%%%%%%%%%%%%%%%%%%%%%%%%%

\recipe{Myrtle Liquer Pudding}
\italianrecipe{Crema al Mirto}
\region{Sardegna}
% And just where are we going to find Myrtle Liquer? Ah, that is 'Mirto' - a Scicilian liquer!

\recipe{Quince and Apple Custard}
\italianrecipe{Flam di Cotogne e Mele}
\vegan
\region{Trentino}
\ingredientredundant{Quinces}
\ingredient{Apples}

\newpage	%377 %%%%%%%%%%%%%%%%%%%%%%%%%%%%%%%%%%%%%%%%%%%%%%%%%%%%%%%%%%%%%%%%%%

\recipe{Coffee Custard}
\italianrecipe{Caff\'{e} in Forchetta}
\region{Toscana}
\ingredient{Eggs}

\recipe{Chocolate Fondue}
\italianrecipe{Fonduta di Cioccolato}
\region{Vale d'Aosta}

\newpage	%378 %%%%%%%%%%%%%%%%%%%%%%%%%%%%%%%%%%%%%%%%%%%%%%%%%%%%%%%%%%%%%%%%%%

\recipe{Chocolate and Squash Pudding}
\italianrecipe{Budino di Zucca e Cioccolato}
\region{Campania}
\ingredient{Squash}

\recipe{Grappa-Flavoured Whipped Cream}
\italianrecipe{Fiocca}
\region{Valle d'Aosta}
\ingredientredundant{Grappa}

\newpage	%379 %%%%%%%%%%%%%%%%%%%%%%%%%%%%%%%%%%%%%%%%%%%%%%%%%%%%%%%%%%%%%%%%%%

\photo{Chocolate and Squash Pudding}

\newpage	%380 %%%%%%%%%%%%%%%%%%%%%%%%%%%%%%%%%%%%%%%%%%%%%%%%%%%%%%%%%%%%%%%%%%

\recipe{Chestnut Mountain}
\italianrecipe{Montebianco}
\region{Valle d'Aosta}
\ingredient{Chestnuts}

\recipe{Polenta with Apples}
\italianrecipe{Polenta e Mele}
\vegan
\region{Alto Adige}
\ingredient{Polenta}
\ingredient{Apples}

\newpage	%381 %%%%%%%%%%%%%%%%%%%%%%%%%%%%%%%%%%%%%%%%%%%%%%%%%%%%%%%%%%%%%%%%%%

\recipe{Grapes Wrapped in Citron Leaves}
\italianrecipe{Panicielli}
\vegan
\region{Calabria}
\ingredientredundant{Grapes}

\recipe{Fig Reduction}
\italianrecipe{Crema di Cotto di Fichi}
\vegan
\region{Basilicata}
\ingredient{Figs}

\newpage	%382 %%%%%%%%%%%%%%%%%%%%%%%%%%%%%%%%%%%%%%%%%%%%%%%%%%%%%%%%%%%%%%%%%%

\recipe{Sour Cherries in Syrup}
\italianrecipe{Amarene Sciroppate}
\vegan
\region{Molise}
\ingredient{Cherries}

\recipe{Lemon and Basil Sorbet}
\italianrecipe{Sorbetto di Limone al Profumo di Basilico}
\vegan
\region{Piemonte}
\ingredient{Lemons}

\newpage	%383 %%%%%%%%%%%%%%%%%%%%%%%%%%%%%%%%%%%%%%%%%%%%%%%%%%%%%%%%%%%%%%%%%%

\photo{Sour Cherries in Syrup}

\newpage	%384 %%%%%%%%%%%%%%%%%%%%%%%%%%%%%%%%%%%%%%%%%%%%%%%%%%%%%%%%%%%%%%%%%%

\recipe{Berry Frozen Yoghurt}
\italianrecipe{Gelato di More e Mirtilli}
\vegan
\region{Lombardia}
\ingredient{Blackberries}

\recipe{Apricot Jam Tart}
\italianrecipe{Crostata con Confettura di Albicocche}
\vegan
\region{Piemonte}
% No leap out vital ingredients?

\newpage	%385 %%%%%%%%%%%%%%%%%%%%%%%%%%%%%%%%%%%%%%%%%%%%%%%%%%%%%%%%%%%%%%%%%%

\recipe{Mascarpone and Nougat Semifreddo Drizzled with Vincotto}
\italianrecipe{Semifreddo al Torroncino con Vincotto}
\region{Basilicata}
% Unclear which ingredients, if any, to list as key?

\newpage	%386 %%%%%%%%%%%%%%%%%%%%%%%%%%%%%%%%%%%%%%%%%%%%%%%%%%%%%%%%%%%%%%%%%%

\recipe{Honey Semifreddo with Apricot Puree}
\italianrecipe{Semifreddo al Miele con Purea di Albicocche}
\region{Friuli Venezia Giulia}
\ingredient{Honey}

\newpage	%387 %%%%%%%%%%%%%%%%%%%%%%%%%%%%%%%%%%%%%%%%%%%%%%%%%%%%%%%%%%%%%%%%%%

\photo{Honey Semifreddo with Apricot Puree}

\newpage	%388 %%%%%%%%%%%%%%%%%%%%%%%%%%%%%%%%%%%%%%%%%%%%%%%%%%%%%%%%%%%%%%%%%%

\recipe{Custard Tart with Hazelnuts}
\italianrecipe{Crostata alla Crema con Nocciole}
\region{Piemonte}
\ingredient{Nuts, Hazelnuts}

\newpage	%389 %%%%%%%%%%%%%%%%%%%%%%%%%%%%%%%%%%%%%%%%%%%%%%%%%%%%%%%%%%%%%%%%%%

\photo{Custard Tart with Hazelnuts}

\newpage	%390 %%%%%%%%%%%%%%%%%%%%%%%%%%%%%%%%%%%%%%%%%%%%%%%%%%%%%%%%%%%%%%%%%%

\recipe{Hot Pepper and Orange Tart}
\italianrecipe{Crostata del Diavolo}
\region{Calabria}
\ingredient{Nuts, Almonds}
% Hidden recipe...
\refcurrentrecipe{Chili Pepper Preserves}

\recipe{Crumbly Sand Tart}
\italianrecipe{Torta Sabbiosa}
\region{Trentino}
\ingredient{Nuts, Almonds}

\newpage	%391 %%%%%%%%%%%%%%%%%%%%%%%%%%%%%%%%%%%%%%%%%%%%%%%%%%%%%%%%%%%%%%%%%%

\recipe{Greek Almond Tart}
\italianrecipe{Torta Greca}
\region{Veneto}
\ingredient{Nuts, Almonds}

\newpage	%392 %%%%%%%%%%%%%%%%%%%%%%%%%%%%%%%%%%%%%%%%%%%%%%%%%%%%%%%%%%%%%%%%%%

\recipe{Apple Strudel}
\italianrecipe{Strudel di Mele}
\region{Trentino}
\ingredient{Apples}
\refcurrentrecipe{Dough, Strudel}

\newpage	%393 %%%%%%%%%%%%%%%%%%%%%%%%%%%%%%%%%%%%%%%%%%%%%%%%%%%%%%%%%%%%%%%%%%

\photo{Apple Strudel}

\newpage	%394 %%%%%%%%%%%%%%%%%%%%%%%%%%%%%%%%%%%%%%%%%%%%%%%%%%%%%%%%%%%%%%%%%%

\recipe{Raspberry Tartlets}
\italianrecipe{Crostatine con Lamponi}
\region{Lazio}
\ingredientredundant{Raspberries}
\refcurrentrecipe{Dough, Lemon}

\newpage	%395 %%%%%%%%%%%%%%%%%%%%%%%%%%%%%%%%%%%%%%%%%%%%%%%%%%%%%%%%%%%%%%%%%%

\photo{Raspberry Tartlets}

\newpage	%396 %%%%%%%%%%%%%%%%%%%%%%%%%%%%%%%%%%%%%%%%%%%%%%%%%%%%%%%%%%%%%%%%%%

\recipe{Crisp Olive Oil Cake}
\italianrecipe{Stroscia}
\region{Liguria}

\recipe{Apple and Pear Cake}
\italianrecipe{Torta di Mele e Pere}
\vegan
\region{Piemonte}
\ingredient{Apples}
\ingredient{Pears}

\newpage	%397 %%%%%%%%%%%%%%%%%%%%%%%%%%%%%%%%%%%%%%%%%%%%%%%%%%%%%%%%%%%%%%%%%%

\recipe{Squash Cake}
\italianrecipe{Torta di Zucca}
\vegan
\region{Veneto}
\ingredient{Squash}

\recipe{Carrot Cake}
\italianrecipe{Torta di Carote}
\region{Veneto}
\ingredient{Carrots}

\newpage	%398 %%%%%%%%%%%%%%%%%%%%%%%%%%%%%%%%%%%%%%%%%%%%%%%%%%%%%%%%%%%%%%%%%%

\recipe{Nut Cake}
\italianrecipe{Krasko Pecivo}
\region{Fruili Venezia Giulia}
\ingredient{Nuts}	% A variety of, so leave unspecified

\recipe{Almond and Lemon Cake}
\italianrecipe{Torta Rosata}
\region{Puglia}
\ingredient{Nuts, Almonds}
\ingredient{Lemons}

\newpage	%399 %%%%%%%%%%%%%%%%%%%%%%%%%%%%%%%%%%%%%%%%%%%%%%%%%%%%%%%%%%%%%%%%%%

\recipe{Flourless Hazelnut Cake}
\italianrecipe{Torta di Nocciole}
\region{Campania}
\ingredient{Nuts, Hazelnuts}

\recipe{Chocolate Almond Cake}
\italianrecipe{Torta Caprese}
\region{Campania}
\ingredient{Nuts, Almonds}

\newpage	%400 %%%%%%%%%%%%%%%%%%%%%%%%%%%%%%%%%%%%%%%%%%%%%%%%%%%%%%%%%%%%%%%%%%

\recipe{Drunken Cake}
\italianrecipe{Schiaccia Briaca}
\region{Toscana}
\ingredient{Nuts}

\newpage	%401 %%%%%%%%%%%%%%%%%%%%%%%%%%%%%%%%%%%%%%%%%%%%%%%%%%%%%%%%%%%%%%%%%%

\recipe{Individual Blueberry Fig Cakes}
\italianrecipe{Muffin di Mirtilli con Fichi}
\region{Lombardia}
\vegan
\ingredient{Blueberries}
\ingredient{Figs}

\recipe{Lingonberry Jam Buckwheat Layer Cake}
\italianrecipe{Torta di Grano Sarceno con Confettura di Mirtilli}
\region{Trentino}
\ingredient{Buckwheat Flour}
\ingredient{Nuts, Almonds}

\newpage	%402 %%%%%%%%%%%%%%%%%%%%%%%%%%%%%%%%%%%%%%%%%%%%%%%%%%%%%%%%%%%%%%%%%%

\recipe{Rice Flour Cake with Strawberries and Vanilla Mouse}
\italianrecipe{Torta di Riso alle Fragole con Mousse alla Vaniglia}
\region{Veneto}
\ingredient{Vanilla}
\ingredient{Strawberries}

\newpage	%403 %%%%%%%%%%%%%%%%%%%%%%%%%%%%%%%%%%%%%%%%%%%%%%%%%%%%%%%%%%%%%%%%%%

\photo{Rice Flour Cake with Strawberries and Vanilla Mouse}

\newpage	%404 %%%%%%%%%%%%%%%%%%%%%%%%%%%%%%%%%%%%%%%%%%%%%%%%%%%%%%%%%%%%%%%%%%

\recipe{Polenta Cake}
\italianrecipe{Amor Polenta}
\region{Umbria}
\ingredient{Nuts, Almonds}
\ingredient{Polenta}

\recipe{Sweet Pancake}
\italianrecipe{Laciada}
\region{Lombardia}
% No standout ingredients - flour, oil, sugar, salt...

\newpage	%405 %%%%%%%%%%%%%%%%%%%%%%%%%%%%%%%%%%%%%%%%%%%%%%%%%%%%%%%%%%%%%%%%%%

\photo{Polenta Cake}

\newpage	%406 %%%%%%%%%%%%%%%%%%%%%%%%%%%%%%%%%%%%%%%%%%%%%%%%%%%%%%%%%%%%%%%%%%

\recipe{Baked Chocolate Polenta}
\italianrecipe{Dolce di Granturco al Cacao}
\region{Lombardia}
\ingredient{Polenta}
\ingredient{Cocoa Powder}

\recipe{Wine and Anise Seed Cookies}
\italianrecipe{Tisichelle}
\vegan
\region{Abruzzo}
\ingredient{Anise Seeds}
\ingredient{Wine, Trebbiano}

\newpage	%407 %%%%%%%%%%%%%%%%%%%%%%%%%%%%%%%%%%%%%%%%%%%%%%%%%%%%%%%%%%%%%%%%%%

\recipe{Dried Fruit and Nut Roll}
\italianrecipe{Gubana}
\region{Friuli Venezia Giulia}
\ingredient{Figs, Dried}
\ingredient{Prunes}
\ingredient{Nuts}	% A variety

\newpage	%408 %%%%%%%%%%%%%%%%%%%%%%%%%%%%%%%%%%%%%%%%%%%%%%%%%%%%%%%%%%%%%%%%%%

\recipe{Layered Pear Cake}
\italianrecipe{Dolce di Pere in Sfoglia}
\region{Lombardia}
\ingredient{Pears}

\newpage	%409 %%%%%%%%%%%%%%%%%%%%%%%%%%%%%%%%%%%%%%%%%%%%%%%%%%%%%%%%%%%%%%%%%%

\photo{Layered Pear Cake}

\newpage	%410 %%%%%%%%%%%%%%%%%%%%%%%%%%%%%%%%%%%%%%%%%%%%%%%%%%%%%%%%%%%%%%%%%%

\recipe{Tile Cookies}
\italianrecipe{Tegole}
\region{Valle d'Aosta}
\ingredient{Nuts}

\recipe{Nut and Raisin Cookies}
\italianrecipe{Balocchi}
\vegan
\region{Lazio}
\ingredient{Nuts}
\ingredient{Raisins}

\newpage	%411 %%%%%%%%%%%%%%%%%%%%%%%%%%%%%%%%%%%%%%%%%%%%%%%%%%%%%%%%%%%%%%%%%%

\photo{Tile Cookies}

\newpage	%412 %%%%%%%%%%%%%%%%%%%%%%%%%%%%%%%%%%%%%%%%%%%%%%%%%%%%%%%%%%%%%%%%%%

\recipe{Almond Cookies}
\italianrecipe{Amaretti}
\region{Liguria}
\ingredient{Nuts, Almonds}
\ingredient{Nuts, Hazelnuts}

\recipe{Meringue Kisses with Almonds}
\italianrecipe{Spumini alle Mandorle}
\region{Marche}
\ingredient{Nuts, Almonds}

\newpage	%413 %%%%%%%%%%%%%%%%%%%%%%%%%%%%%%%%%%%%%%%%%%%%%%%%%%%%%%%%%%%%%%%%%%

\recipe{Cornmeal Cookies}
\italianrecipe{Becc\'{u}te}
\vegan
\region{Marche}
\ingredient{Cornmeal}

\recipe{Strudel Cookies}
\italianrecipe{Biscotti Strudel}
\vegan
\region{Emilia-Romagna}
% Nothing really to note

\newpage	%414 %%%%%%%%%%%%%%%%%%%%%%%%%%%%%%%%%%%%%%%%%%%%%%%%%%%%%%%%%%%%%%%%%%

\recipe{Wine Ring Cookies}
\italianrecipe{Ciambelline con il Vino}
\vegan
\region{Calabria}
\ingredient{Semolina}

\recipe{Sweet Fritters}
\italianrecipe{Bracalaccio}
\region{Umbria}
\ingredient{Semolina}

\newpage	%415 %%%%%%%%%%%%%%%%%%%%%%%%%%%%%%%%%%%%%%%%%%%%%%%%%%%%%%%%%%%%%%%%%%

\recipe{Crescent Pastries}
\italianrecipe{Chiffel}
\region{Friuli Venezia Giulia}

\newpage	%416 %%%%%%%%%%%%%%%%%%%%%%%%%%%%%%%%%%%%%%%%%%%%%%%%%%%%%%%%%%%%%%%%%%

\recipe{Apricot Potato Dumplings}
\italianrecipe{Kn\"{o}del di Albicocca}
\region{Alto Adige}
\ingredient{Potatoes}
\ingredientredundant{Apricots}

\newpage	%417 %%%%%%%%%%%%%%%%%%%%%%%%%%%%%%%%%%%%%%%%%%%%%%%%%%%%%%%%%%%%%%%%%%

\photo{Apricot Potato Dumplings}

\newpage	%418 %%%%%%%%%%%%%%%%%%%%%%%%%%%%%%%%%%%%%%%%%%%%%%%%%%%%%%%%%%%%%%%%%%

\recipe{Baba au Rum}
\italianrecipe{Bab\`{a} al rum}
\refcurrentrecipe{Rum Baba}	% As known here...
\region{Campania}

\newpage	%419 %%%%%%%%%%%%%%%%%%%%%%%%%%%%%%%%%%%%%%%%%%%%%%%%%%%%%%%%%%%%%%%%%%

\recipe{Doughnuts}
\italianrecipe{Zeppole}
\region{Basilicata}
\ingredient{Potatoes}	% Yes, really!
\ingredient{Semolina}

\newpage	%420 %%%%%%%%%%%%%%%%%%%%%%%%%%%%%%%%%%%%%%%%%%%%%%%%%%%%%%%%%%%%%%%%%%

\recipe{Rosemary Buns}
\italianrecipe{Pan di Ramerino}
\region{Toscana}

\newpage	%421 %%%%%%%%%%%%%%%%%%%%%%%%%%%%%%%%%%%%%%%%%%%%%%%%%%%%%%%%%%%%%%%%%%

\photo{Rosemary Buns}

\newpage	%422 %%%%%%%%%%%%%%%%%%%%%%%%%%%%%%%%%%%%%%%%%%%%%%%%%%%%%%%%%%%%%%%%%%

\recipe{Fried Dough Rosettes}
\italianrecipe{Cartellate}
\region{Puglia}

\newpage	%423 %%%%%%%%%%%%%%%%%%%%%%%%%%%%%%%%%%%%%%%%%%%%%%%%%%%%%%%%%%%%%%%%%%

\photo{Fried Dough Rosettes}

\newpage	%424 %%%%%%%%%%%%%%%%%%%%%%%%%%%%%%%%%%%%%%%%%%%%%%%%%%%%%%%%%%%%%%%%%%

\recipe{Cr\^{e}pes with Ice Cream}
\italianrecipe{Cr\^{e}pe con Gelato}
\region{Emilia-Romagna}

\recipe{Apple Fritters}
\italianrecipe{Frittelle di Mele}
\region{Valle d'Aosta}
\ingredient{Apples}

\newpage	%425 %%%%%%%%%%%%%%%%%%%%%%%%%%%%%%%%%%%%%%%%%%%%%%%%%%%%%%%%%%%%%%%%%%

\photo{Cr\^{e}pes with Ice Cream}

\newpage	%426 %%%%%%%%%%%%%%%%%%%%%%%%%%%%%%%%%%%%%%%%%%%%%%%%%%%%%%%%%%%%%%%%%%

\recipe{Bread and Jam Fritters}
\italianrecipe{Frittelle di Pane}
\region{Trentino}

\recipe{Chestnut Flour Fritters}
\italianrecipe{Frittelle di Farina di Castagne}
\vegan
\region{Toscana}
\ingredient{Chestnut Flour}

\newpage	%427 %%%%%%%%%%%%%%%%%%%%%%%%%%%%%%%%%%%%%%%%%%%%%%%%%%%%%%%%%%%%%%%%%%

\recipe{Chocolate Buckwheat Fritters}
\italianrecipe{Sciatt al Cioccolato}
\vegan
\region{Lombardia}
\ingredient{Buckwheat Flour}
\ingredient{Chocolate, Dark}

\recipe{Quince Diamonds}
\italianrecipe{Cotognata}
\vegan
\region{Liguria}
\refcurrentrecipe{Membrillo}	% The more common Spanish name...
\ingredientredundant{Quinces}

\newpage	%428 %%%%%%%%%%%%%%%%%%%%%%%%%%%%%%%%%%%%%%%%%%%%%%%%%%%%%%%%%%%%%%%%%%

\recipe{Cheese Turnovers with Honey}
\italianrecipe{Seadas}
\region{Sardegna}
\ingredient{Cheese, Sheep's}

\recipe{Fig and Chocolate Kisses}
\italianrecipe{Baci di Fichi e Cioccolato}
\vegan
\region{Calabria}
\ingredient{Figs}
\ingredient{Nuts, Almonds}
\ingredient{Chocolate}

\newpage	%429 %%%%%%%%%%%%%%%%%%%%%%%%%%%%%%%%%%%%%%%%%%%%%%%%%%%%%%%%%%%%%%%%%%

\recipe{Cannoli with Ricotta Filling}
\italianrecipe{Cannoli di Ricotta}
\region{Sicilia}
\ingredient{Cheese, Ricotta}

\newpage	%430 %%%%%%%%%%%%%%%%%%%%%%%%%%%%%%%%%%%%%%%%%%%%%%%%%%%%%%%%%%%%%%%%%%

\recipe{Meringue Hazelnut Drops}
\italianrecipe{Nocciolini}
\region{Piemonte}
\ingredient{Nuts, Hazelnuts}

\recipe{Chopped Figs Dipped in Dark Chocolate}
\italianrecipe{Fichi e Cioccolato}
\region{Campania}
\ingredient{Figs}
\ingredient{Chocolate, Dark}

\newpage	%431 %%%%%%%%%%%%%%%%%%%%%%%%%%%%%%%%%%%%%%%%%%%%%%%%%%%%%%%%%%%%%%%%%%

\recipe{Fig and Nut Salami}
\italianrecipe{Lonzino di Fico}
\vegan
\region{March}
\ingredient{Figs}
\ingredient{Nuts, Walnuts}
\ingredient{Nuts, Almonds}

\recipe{Prickly Pear Jellies}
\italianrecipe{Gelo di Fichi d'India}
\vegan
\region{Sicilia}
\ingredientredundant{Prickly Pears}

\newpage	%432 %%%%%%%%%%%%%%%%%%%%%%%%%%%%%%%%%%%%%%%%%%%%%%%%%%%%%%%%%%%%%%%%%%

\recipe{Sugar Doves}
\italianrecipe{Colombine}
\vegan
\region{Sicilia}

\newpage	%433 %%%%%%%%%%%%%%%%%%%%%%%%%%%%%%%%%%%%%%%%%%%%%%%%%%%%%%%%%%%%%%%%%%

\photo{Sugar Doves}

\newpage	%434 %%%%%%%%%%%%%%%%%%%%%%%%%%%%%%%%%%%%%%%%%%%%%%%%%%%%%%%%%%%%%%%%%%

\recipe{Pompia and Almond Squares}
\italianrecipe{Aranzata de Pompia}
\region{Sicilia}
\ingredientredundant{Pompia}	% A Sour citrus from Sardegna
\ingredient{Nuts, Almonds}

\newpage	%435 %%%%%%%%%%%%%%%%%%%%%%%%%%%%%%%%%%%%%%%%%%%%%%%%%%%%%%%%%%%%%%%%%%

\recipe{Quince Cubes}
\italianrecipe{Persegada}
\vegan
\region{Veneto}
\ingredientredundant{Quinces}

%%%%%%%%%%%%%%%%%%%%%%%%%%%%% END OF DESSERTS %%%%%%%%%%%%%%%%%%%%%%%%%%%%%%%%%%%%%


% %%%%%%%%%%%%%%% END OF DOCUMENT %%%%%%%%%%%%%%%%%%

\printindex
\end{document}

